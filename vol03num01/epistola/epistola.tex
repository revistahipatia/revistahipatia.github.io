\documentclass[onecolumn]{hipatia}
\usepackage{blindtext}
\newcommand{\superau}{\textsuperscript{\underline{a}}~}
\title{Realizações}
\subtitle{Epístola}
\author{}
\begin{document}
\setcounter{page}{\epistolapage}
\maketitle
\leftskip=2.5cm
\rightskip=2.5cm


\noindent Caro Leitor,
\vspace{1cm}

É com grande satisfação que compartilhamos 
os avanços recentes na consolidação da 
Revista de Matemática Hipátia desde nossa 
última edição. Obtivemos o código ISSN 
junto ao Instituto Brasileiro de Informação 
em Ciência e Tecnologia (IBICT), criamos 
uma conta no repositório Zenodo para 
registro de Identificadores de Objeto Digital 
(DOI), garantindo \emph{links} permanentes para 
nossos artigos, e fomos incluídos no 
diretório do \emph{Latindex}, 
sistema de indexação de periódicos voltado 
para a América Latina, Caribe, Portugal e 
Espanha, sediado na Universidade Nacional 
Autônoma do México (UNAM). 

Essas conquistas ampliam o alcance da 
revista e permitem que nossos artigos 
sejam reconhecidos como publicações 
acadêmicas em diversos contextos, 
como progressões funcionais, exames de 
qualificação e avaliações de órgãos de 
fomento. Com isso, reforçamos nossa 
responsabilidade de oferecer conteúdo 
de excelência aos nossos leitores.

Nesta edição, apresentamos uma rica 
diversidade de temas. Na seção \textsc{História}, 
abordo brevemente a relevância dos \emph{Elementos} de 
Euclides, comparando-o a um farol --- 
como o Farol de Alexandria, que ilustra nossa CAPA ---
por sua luz orientadora na matemática. 
A seção seguinte, intitulada \textsc{Biografia}, 
merece uma explicação especial. Ao saber 
que o professor Dionicarlos preparava um 
livro sobre sua esposa, a estimada professora 
Elinalva, decidimos incluir um excerto 
dessa obra, com a colaboração da professora 
Elaís Cidely, uma de nossas editoras. 
A profusão de depoimentos elogiosos e merecidos 
à professora Elinalva 
--- aos quais me junto com meus 
próprios elogios ---
pode dar uma ideia, 
mesmo para aqueles que não a conheceram, 
da alegria e do entusiasmo que marcaram 
sua trajetória.

\pagebreak

Merecem destaque também os professores 
Samuel e Diego, que, na seção \textsc{Teorema}, 
conduzem um brilhante \emph{tour de force} em 
lógica matemática, demonstrando a conexão 
entre duas proposições aparentemente 
distintas: a Hipótese Generalizada do Contínuo 
e o Axioma da Escolha. 

Pela primeira vez, introduzimos a seção 
\textsc{Memória}, alinhada ao objetivo da 
Revista de Matemática Hipátia de preservar 
a história do Departamento de Matemática 
da UFBA e da matemática na Bahia e no Brasil. 
Nela, o professor José Fernandes compartilha 
um depoimento pessoal, oferecendo um valioso 
relato de sua trajetória na matemática. 
Encerramos, como de costume, com a seção \textsc{Problema}, 
convidando os leitores a novos desafios.



\vspace{1cm}

\hfill Salvador, 13 de julho de 2025.

\hfill O Editor
\end{document}
