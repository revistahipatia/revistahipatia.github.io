\RequirePackage[hyphens]{url}
\documentclass{hipatia}
\usepackage{lipsum}
%Use \DeclareMathOperator para definir novos
% operadores para o modo matemático
\DeclareMathOperator{\sen}{sen}

%Evite numerar teoremas
%Prefira nomeá-los
%Use os ambientes abaixo
\newtheorem*{theorem*}{Teorema}
\newtheorem*{lemma*}{Lema}

% Evite títulos muito longos
\title{Composições Geométricas:\\
Unindo Arte e Matemática}
% Se for necessário diminuir a fonte do título 
% para caber no quadro, use
% \title{ \fontsize{28}{28}\selectfont Uma Nova Demonstração do\\  \fontsize{28}{28}\selectfont Teorema de Pitágoras}

% O Subtítulo é o nome da seção da revista
% Deve ser uma palavra de origem grega
\subtitle{Didática}
\author{Maiara Santos e Vinícius Mello}
% A data não é necessária
%\date{October 2023}
% Não se preocupe com a numeração
% das páginas ou com o número da edição

\begin{document}
\setcounter{page}{\didaticapage}
\maketitle

\section{Introdução}

Ao ensinar Matemática na educação básica,
podemos nos deparar 
com questionamentos por parte dos discentes 
sobre a utilidade ou aplicação prática
dos temas estudados. Esses questionamentos 
podem indicar a necessidade de utilizar 
abordagens em sala de aula que mostrem a 
Matemática de forma integrada ou sua relação 
com outras áreas do conhecimento. Isso pode 
ser feito ao ensinarmos tópicos da  
Matemática numa perspectiva interdisciplinar, 
no intuito de  expor a relevância de 
conteúdos matemáticos associados a temáticas 
de outras disciplinas e também provocar 
o interesse dos estudantes para o estudo 
da Matemática.
%Dentre as várias formas de praticar a 
%interdisciplinaridade, nessa dissertação, 
%escolhemos associar o conhecimento matemático 
%geométrico à Arte, com o auxílio da tecnologia. 
Concordamos com \cite{Silva2024} quando 
se afirma que 
a interdisciplinaridade vai permitir uma 
melhor compreensão de fenômenos e problemas 
complexos, além de contribuir para a redução da 
``fragmentação do conhecimento e a 
limitação das disciplinas básicas, 
ao estimular o diálogo entre diferentes 
campos e perspectivas''. 

Neste trabalho, que resume a
pesquisa desenvolvida na dissertação de 
mestrado de Maiara Santos pelo PROFMAT,
Programa de Mestrado Profissional em 
Matemática em Rede Nacional,
sob a orientação
do professor Vinícius Mello,
a Arte surge como uma motivadora 
para o ensino e a aprendizagem da Matemática. 
A escolha pela área das artes visuais se 
deu pela 
possibilidade de associá-la ao conhecimento 
matemático através da \emph{geometria}. 
Acreditamos que as artes visuais podem 
contribuir para a percepção da Matemática 
como parte do cotidiano e como produção humana, 
além de funcionar como ponto de partida para 
discussões mais empíricas sobre o conhecimento 
matemático na sala de aula. 
Como bem resumido por \cite{Mendes2012}, 
\begin{quote}
Ao longo da sua existência, a sociedade humana 
construiu uma variedade cultural que se 
manifesta por meio de atividades relacionadas 
à arte e que podem ser interpretadas como uma 
aplicação de conceitos e técnicas geométricas, 
principalmente aquelas cujos princípios 
geométricos são centrais na construção de 
um desenho ou projeto artístico.
\end{quote}

Dentro das artes visuais, podemos observar a 
presença da matemática com diferentes 
níveis de complexidade e intencionalidade. 
Existem movimentos artísticos que de forma 
intencional utilizam os conceitos matemáticos 
e a abstração por eles fornecida, juntamente 
com uso das cores, para expressar ideias
em obras de arte. 
Chamaremos a organização das cores e 
conceitos geométricos observados nessas 
pinturas de \emph{composição geométrica}. 
Assim, nos interessa estudar a influência 
que essas composições podem ter no ensino 
de conteúdos matemáticos. 

Quanto a presença da matemática no cotidiano 
do estudante, resolvemos desenvolver uma proposta 
interdisciplinar tendo como meio de desenvolvimento 
um \emph{software} livre de geometria dinâmica,
o GeoGebra. 
As ferramentas tecnológicas fazem parte do 
dia a dia da nossa sociedade e existem metodologias 
que incentivam o uso de diferentes tecnologias 
em sala de aula. Acreditamos que as ferramentas 
tecnológicas podem ser grandes aliadas no 
ensino de matemática, quando utilizadas de 
forma organizada, didática e com viés pedagógico, 
visando os processos de ensino e de aprendizagem. 

O tema da nossa pesquisa, Composições Geométricas 
no GeoGebra, surge, portanto, da percepção
que realizar uma pesquisa 
envolvendo geometria, arte e tecnologia para a 
educação básica poderia trazer bons resultados 
relacionados ao ensino de matemática.
Nessa perspectiva, com o intuito de unir 
matemática, arte e tecnologia, construímos a 
seguinte questão de  pesquisa: \emph{de que forma o uso 
de composições geométricas, aliadas à tecnologia, 
pode favorecer o ensino de conteúdos matemáticos 
geométricos?}

Assim, nosso objetivo de pesquisa é analisar 
as contribuições para estudantes do ensino médio 
de sequências didáticas sobre composições 
geométricas no GeoGebra para o ensino de 
conhecimentos geométricos.
% Estabelecemos 
%como objetivos específicos: 
%\begin{itemize}
%\item Desenvolver sequências didáticas que 
%utilizem composições geométricas para o 
%ensino de conteúdos matemáticos;
%\item Apresentar o desenvolvimento de uma 
%oficina sobre Composições Geométricas com o 
%auxílio do GeoGebra;
%\item Investigar a percepção dos estudantes as 
%sequências didáticas e sobre a oficina ao 
%final das oficinas;
%\item Entender a viabilidade do ensino de 
%conteúdos matemáticos a partir da 
%interdisciplinaridade com a arte.
%\end{itemize}
Neste artigo para a Revista de Matemática 
Hipátia, resumiremos nossa pesquisa, começando por
delinear o conceito de 
``composição geométrica'', através de seus princípios, 
elementos, desenvolvimento histórico e 
principais expoentes, e terminando
com a descrição de sua metodologia, execução e resultados.

\section{Composição: Elementos e Princípios}

O termo composições geométricas ocupa o centro das
discussões, por ser o fator de interdisciplinaridade 
entre Matemática e
Arte que gera os nossos estudos. Estamos explorando a 
potencialidade das
composições geométricas para mostrar a matemática 
existente em obras de
arte e é necessário compreendermos o que essa expressão 
significa. Iniciaremos tentando compreender o que o
termo `composição' significa nas
artes visuais.

Ao buscarmos em dicionário, a palavra composição apresenta vários
significados, dos quais enfatizamos: ``Ação ou efeito de compor, formar
um todo. Disposição do que constitui um todo; constituição. Maneira como
algo está ou se encontra disposto; organização. {[}Artes Plásticas{]}
Constituição ou desenvolvimento da estrutura da obra de arte''
\cite{dicionario}. No contexto da nossa pesquisa, adotamos 
uma definição inspirada pela dada em \cite{joseph}:
 \emph{composição é a organização dos elementos visuais
 para comunicar uma intenção}.
A palavra-chave aqui é intenção, pois a organização dos elementos
visuais não é aleatória, mas sim planejada pelo
artista para expressar
sentidos, emoções, sensações e, no caso das \emph{composições geométricas},
conceitos matemáticos.

\begin{figure*}[htb!]
	\centering
	\includegraphics[width=0.8\textwidth]{ElComp.png}
	\caption{Elementos da composição. Figura produzida 
	apenas com recursos do GeoGebra.}
	\label{fig:ElComp}
\end{figure*}

Para completar essa definição, entretanto, precisamos entender o que são 
precisamente os \emph{elementos} visuais e em que 
\emph{princípios} se baseia essa 
organização. Como não há um consenso na literatura sobre 
quais sejam esses elementos e princípios,
até pela dificuldade de se estabelecer definições precisas para
conceitos tão subjetivos, optamos por adaptar definições
encontradas em \cite{dondis} e \cite{gatto} para o nosso contexto.

Os elementos da composição são:
\begin{description}
	\item[Linha] É considerada o elemento fundamental da composição, 
	definida como o rastro de um ponto em movimento. As linhas podem 
	ser retas, curvas, finas ou grossas e servem para criar movimento, 
	dar direção e guiar o olhar do espectador através da obra.
	\item[Forma] Refere-se a espaços fechados criados quando linhas 
	se encontram ou áreas de cor e textura se intersectam. 
	Elas podem ser geométricas (como quadrados e círculos) ou 
	orgânicas (semelhança com seres vivos). Na composição, a 
	forma serve como um elemento unificador que vincula os outros 
	componentes. 
	\item[Volume (ou Forma Tridimensional)] 
	É a técnica de fazer objetos parecerem tridimensionais em uma 
	superfície plana. Isso é alcançado através do uso de perspectiva, 
	luz e sombra, criando a ilusão de profundidade. No espaço real, 
	como na escultura, o volume é tangível, enquanto na pintura ele 
	é uma representação visual de massa e peso que influencia o 
	equilíbrio da composição.
	\item[Cor] É um dos elementos mais expressivos, capaz de criar 
	harmonia, contraste e profundidade sem o uso de palavras. 
	A cor atrai a atenção para partes específicas da obra e atua 
	como símbolo ou metáfora (ex: azul para serenidade, vermelho 
	para paixão). O efeito de uma cor é sempre relativo à sua 
	situação em relação às cores vizinhas.
	\item[Valor (ou Tom)] Refere-se ao grau de claridade ou 
	obscuridade de uma cor ou superfície. É essencial para criar 
	contraste e profundidade, permitindo que o artista 
	destaque pontos de luz intensa ou sombras profundas. O valor 
	dá à pintura seu senso de realismo e dimensão.
	\item[Textura] Diz respeito à qualidade da superfície ou à 
	sensação tátil de um objeto, podendo ser percebida visualmente 
	ou sentida fisicamente. Ela adiciona uma dimensão sensorial 
	à composição.
	\item[Espaço] Refere-se às áreas ao redor e entre os objetos.
	 Inclui o espaço positivo (ocupado pelos elementos principais)
	  e o espaço negativo (as áreas abertas ao redor deles). 
		O espaço é vital para definir limites, escala e criar ênfase 
		na composição. Em artes monumentais, o espaço também 
		envolve o movimento físico do espectador para apreciar 
		a obra de diferentes pontos de vista.
\end{description}
Estes elementos estão ilustrados na Figura~\ref{fig:ElComp},
que foi produzida utilizando apenas recursos do GeoGebra.

Já os princípios da composição estão ligados à forma como os
elementos visuais se relacionam para criar uma obra coesa e
expressiva. São eles:
\begin{description}
	\item[Equilíbrio (ou Harmonia)]
O equilíbrio refere-se à distribuição igualitária de 
elementos visuais dentro de uma composição. 
Ele é essencial para conferir estabilidade e unidade à obra, 
podendo ser alcançado de dois modos principais:
\emph{simétrico}, quando as partes são organizadas de 
forma idêntica ou espelhada em relação a um eixo, geralmente vertical,
e \emph{assimétrico} quando elementos diferentes 
(em peso visual, cor ou forma) são organizados de modo a 
contrabalançar uns aos outros, criando uma estabilidade sem repetição exata.
	\item[Proporção]
Este princípio trata da relação de tamanho entre objetos 
ou partes de um todo. A proporção é vital para 
estabelecer harmonia e unidade. Historicamente, artistas 
utilizaram sistemas matemáticos para garantir proporções ``perfeitas'', 
como a \emph{proporção áurea} ($1:\phi$, com $\phi=\frac{1+\sqrt{5}}{2}$)
 e os \emph{retângulos de raiz}($1:\sqrt{2}$, $1:\sqrt{3}$ etc.)
 (Fig.~\ref{fig:retraiz}).
	\item[Contraste] O contraste ocorre quando se percebem 
	diferenças distintas entre dois efeitos comparados. 
	Ele serve para intensificar ou enfraquecer o impacto visual 
	e é crucial para criar hierarquia. 
	\item[Ênfase] A ênfase é o princípio de tornar certos 
	elementos mais proeminentes para criar um ponto focal que 
	capture a atenção do espectador. Isso é geralmente alcançado 
	através do contraste de cor, valor ou nitidez de bordas. 
	\item[Movimento] O movimento é a ilusão de ação dentro de uma 
	composição estática. Ele guia o olhar do espectador através da 
	obra e pode ser gerado pela repetição de linhas, curvas ou diagonais. 
	\item[Ritmo] O ritmo é criado pela repetição de elementos 
	visuais (como cores, formas ou linhas) para produzir um 
	senso de fluxo e energia. Ele funciona de forma análoga 
	à música, apresentando uma sucessão de valores no tempo e 
	no espaço. O ritmo ajuda a organizar o olhar do observador, 
	podendo evocar sentimentos de calma ou excitação.
	\item[Unidade] A unidade refere-se a como todos os 
	elementos são colocados juntos para criar um senso de 
	ordem e completude. A geometria é frequentemente a 
	ferramenta usada para conferir essa unidade estrutural 
	invisível à obra.
\end{description}

Esses princípios serão referenciados ao longo do artigo
quando discutirmos as composições apresentadas.


\begin{figure*}[h]
	\centering
	\includegraphics[width=0.8\textwidth]{descenso.jpg}
	\caption{\emph{A Deposição da Cruz}, de Rogier van der Weyden,
	Museu do Prado, com a composição geométrica proposta
	por Bouleau sobreposta.}
	\label{fig:descenso}
\end{figure*}

\section{Histórico e Principais Expoentes}


Até o início do século XX, a composição geométrica
era um recurso utilizado principalmente
na fase preparatória de obras artísticas,
através de esboços e estudos. Em seu monumental tratado
\emph{La Géometrie Secréte des Peintres} 
(``A Geometria Secreta dos Pintores'', \cite{bouleau}),
o pintor e teórico Charles Bouleau analisa centenas
de obras de arte, desde a antiguidade até o século XX,
e demonstra como os artistas utilizavam
recursos geométricos para estruturar suas composições.
Traremos apenas dois exemplos dessa prática, retirados
do livro de Bouleau.


O primeiro exemplo é a pintura \emph{A Deposição da Cruz},
de Rogier van der Weyden (Fig.~\ref{fig:descenso}).
Bouleau observa que o quadro possui um formato incomum, 
assemelhando-se a um tríptico amalgamado em um único bloco, 
mantendo a simetria e a composição ternária herdada dos 
retábulos de abas. A construção da moldura baseia-se em
um retângulo de proporção $1: \sqrt{3}$ (Fig. \ref{fig:retraiz}),
 mas o que mais chama a atenção
é a complexa estrutura geométrica que organiza a cena central.
O coração da composição, segundo o autor, é organizado por meio
 de um complexo jogo de figuras geométricas. Existem 
 três círculos de mesmo raio, dois deles tangenciais às 
 bordas laterais e ao topo, e o ponto de intersecção entre eles 
 serve como centro para um terceiro círculo central.
Dentro desses círculos, van der Weyden inscreveu pentágonos. 
Bouleau afirma que as diagonais desses pentágonos conferem vigor
 e arquitetura às formas, que de outra forma pareceriam apenas 
 um ``redemoinho'' de corpos. 
	Bouleau ressalta que, apesar desse rigor matemático, 
	a obra atinge um equilíbrio perfeito entre a
	geometria e a emoção. O efeito dramático é intensificado 
	pelas curvas dos corpos suplicantes, que se 
	inclinam em direção a dois polos principais: 
	o Cristo e a Virgem Maria.
As diagonais e os eixos não são apenas auxílios de 
harmonia, mas o objetivo artístico acentuado, onde o 
``humano se sujeita às exigências da geometria''.

\begin{figure}[htb]
	\centering
	\includegraphics[width=7cm]{RetRaiz.png}
	\caption{Construção da razão áurea e dos 
	retângulos de raiz a partir do quadrado. 
	%A título de curiosidade, o papel A4 tem proporção $1:\sqrt{2}$.
	}
	\label{fig:retraiz}
\end{figure}

O outro exemplo, com uma composição bem mais 
simples, é a pintura \emph{Lição de Anatomia do Dr. Tulp},
de Rembrandt (Fig.~\ref{fig:rembrandt}).
Bouleau descreve-a como sendo estruturada a partir de um 
esquema que utiliza quatro pontos, um em cada 
lado da pintura. A partir desses pontos, são traçadas 
linhas inclinadas em direção aos cantos da obra, 
formando dois pares de linhas paralelas. 
Esse arranjo resulta em um paralelogramo que é 
dividido por uma diagonal em dois triângulos iguais.
Dentro dessa organização geométrica, Rembrandt 
agrupa os retratos dos assistentes no triângulo superior. 
Já o cadáver ocupa quase todo o espaço do triângulo inferior. 
Essa estrutura oculta permite a Rembrandt sacrificar 
detalhes secundários 
para enfatizar o centro de interesse 
através da luz, uma técnica que o autor compara ao 
trabalho de um diretor de cena, com os triângulos
direcionando o olhar do espectador dos assistentes para o
o dr. Tulp e para o cadáver.


\begin{figure}[htb]
	\centering
	\includegraphics[width=9cm]{rembrandt.jpg}
	\caption{\emph{Lição de Anatomia do Dr. Tulp}, de Rembrandt,
	Museu Mauritshuis, com a composição geométrica proposta
	por Bouleau sobreposta.}
	\label{fig:rembrandt}
\end{figure}


A partir do século XX, em parte devido ao impacto
da fotografia e do cinema, mas também graças
a busca de novas formas de expressão artística,
a composição geométrica deixa de ser um recurso
utilizado apenas na fase preparatória das obras
e passa a ser o foco principal de diversos movimentos
artísticos. Por exemplo, em \emph{Círculos em um Círculo},
de Wassily Kandinsky (Fig.~\ref{fig:kandinsky}),
a composição é inteiramente baseada em formas geométricas
e cores, organizadas de forma a criar equilíbrio,
movimento e ritmo, sem qualquer representação figurativa.
\begin{figure}[htb]
	\centering
	\includegraphics[width=8	cm]{kandinsky.jpg}
	\caption{\emph{Círculos em um Círculo}, de Wassily Kandinsky,
	Museu Guggenheim.}
	\label{fig:kandinsky}
\end{figure}

Kandinsky, além de ser um dos pioneiros da arte abstrata, foi
professor na Bauhaus, 
escola alemã que funcionou por 14 anos (de 1919 a 1933, 
até ser fechada pelo regime nazista) e deixou
contribuições no que diz respeito a arte, mais 
especificamente a arte
concreta, na arquitetura e no \emph{design} que 
reverberam até os dias atuais.
Em seus fundamentos, utilizava-se da abstração geométrica a 
partir dos
elementos básicos da composição visual, sendo 
referência para o \emph{design}
contemporâneo.

\begin{figure}[htb]
\begin{center}
\includegraphics[width=4.2cm]{bauhaus.png}
\includegraphics[width=4.2cm]{Figuras/image7.png}
\end{center}
\caption{Emblema da Bauhaus, de Oskar Schlemmer e cartaz \emph{Bauhaus Ausstellung} de Fritz Schleifer.}
\label{fig:bauhaus}
\end{figure}

Um  exemplo é o
cartaz de uma exposição da Bauhaus, feita pelo artista 
Fritz Schleifer (Fig.~\ref{fig:bauhaus}), seguindo os princípios \emph{construtivistas} 
que eram defendidos
pela Bauhaus. Elam explica em \cite{elam} 
que o cartaz
apresenta um rosto humano que é representado de forma 
abstrata através
de cinco formas geométricas retangulares, 
uma simplificação do emblema da Bauhaus concebido pelo artista
Oskar Schlemmer, e que há um alinhamento entre o quadrado que
representa um olho e o eixo vertical, mas os demais elementos estão
organizados de forma assimétrica em relação a esse eixo. Além disso, 
percebemos que as formas geométricas estão pintadas com as cores
primárias vermelho e azul.

Após a Segunda Guerra Mundial, outra escola
influente de \emph{design} foi criada na 
Alemanha: a \emph{HfG Ulm} (\emph{Hochschule für Gestaltung Ulm},
ou Escola Superior de \emph{Design} de Ulm), 
fundada em 1953 por ex-integrantes da Bauhaus. 
Vamos citar três expoentes dessa escola que
contribuíram para o desenvolvimento 
da composição geométrica no \emph{design}.

O primeiro deles é o arquiteto e \emph{designer}
Max Bill (1908--1994), que foi aluno da Bauhaus
e um dos fundadores da HfG Ulm. Bill defendia
uma arte baseada na razão e na clareza,
utilizando formas geométricas simples e cores
primárias para criar composições equilibradas
e harmoniosas. Seu trabalho influenciou o \emph{design}
gráfico, industrial e arquitetônico. Um exemplo
de seu trabalho está em uma das sequências didáticas
que desenvolvemos.

O segundo nome é Johannes Itten (1888--1967),
que também foi professor na Bauhaus e teve
influência significativa no desenvolvimento da composição geométrica.
Itten foi um dos primeiros a ensinar a teoria da cor e 
a composição geométrica na Bauhaus, e suas ideias foram 
fundamentais para o movimento
construtivista. 
Em seu livro
\emph{The Elements of Color} \cite{itten}, 
Itten aborda a teoria estética da cor,
resultados de anos de estudo sobre o tema. Ele discute a 
existência da
harmonia das cores que surge a partir do relacionamento entre um
conjunto de matizes. Assim como existe o equilíbrio 
na disposição
desses elementos, há arranjos que não harmonizam e 
podem ter o sentindo
provocativo. A posição, a proporção, o grau de pureza e o 
brilho são
determinantes para a mensagem que se pretende passar em uma 
composição
visual:
\begin{quote}
Uma cor particular incita o olho, por uma sensação específica, 
a buscar
a generalidade. Para, então, buscar essa totalidade, 
para se satisfazer,
o olho busca, além de qualquer espaço de cor, um 
espaço incolor onde
produzir a cor ausente. Aqui temos a regra 
fundamental de toda harmonia
cromática.
\end{quote}
 Ele desenvolveu 
o \emph{círculo cromático} (Fig.~\ref{fig:circulo}), que é um fundamento 
indispensável ao 
tratarmos de teoria
estética da cor.

\begin{figure}
\begin{center}
\includegraphics[width=7cm,trim={0 0.3cm 0.3cm 0.3cm},clip,angle=90]{Figuras/image16.png}
\end{center}
\caption{Círculo cromático de Johannes Itten.}
\label{fig:circulo}
\end{figure}

O terceiro nome é Hermann von Baravalle (1898--1973).
Convidado por Max Bill e Inge Scholl, Baravalle 
atuou como professor convidado regular na \emph{hfg Ulm}
entre 1954 e 1968, integrando o \emph{Grundlehre} 
(Curso Básico).
Sua ``marca'' pedagógica em Ulm foi o ensino da 
\emph{Geometria Dinâmica}, onde as formas não eram estáticas, 
mas vistas como o resultado de tensões e transformações 
originadas pelo movimento (Fig.~\ref{fig:baravalle}).
Diferente da geometria descritiva tradicional, 
focada apenas na representação de objetos, 
Baravalle propunha a geometria como um idioma visual \cite{roldan}. 

\begin{figure}[htb]
	\centering
	\includegraphics[width=8cm]{Baravalle.jpg}
	\caption{Baravalle e suas composições geométricas. Fonte: 
	Revista LIFE 21 mar. 1949.}
	\label{fig:baravalle}
\end{figure}

Antes de sua atuação em Ulm,
Baravalle fora um dos pioneiros do 
Movimento Waldorf, tendo sido escolhido pelo 
próprio Rudolf Steiner --- 
 filósofo e pensador austríaco que desenvolveu a 
 Pedagogia Waldorf, um método educacional baseado na 
 Antroposofia --- para integrar o corpo 
docente da primeira escola em Stuttgart, em 1920.
O objetivo central de Baravalle em ambas as 
instituições era ``reativar o sentido geométrico''. 
Ele acreditava que, ao ensinar os alunos a descobrir 
as leis matemáticas por meio da percepção direta e 
do desenho manual, eles se tornariam criadores 
mais conscientes e autônomos.

A influência da Escola de Ulm no Brasil foi 
profunda e estruturante, atuando como um catalisador 
para a modernização do design gráfico, da arte e 
do ensino superior no país a partir da década de 1950. 
O ponto de partida histórico dessa influência ocorreu em 1951, 
quando a escultura ``Unidade Tripartida'' de Max Bill 
foi premiada na Primeira Bienal de São Paulo, tornando-se 
o marco de apoio para o movimento concretista brasileiro 
ao introduzir o rigor matemático e o anti-romantismo na 
produção nacional.
No campo educacional, a Escola de Ulm serviu de modelo direto 
para a criação da ESDI (Escola Superior de Desenho Industrial) 
no Rio de Janeiro, em 1963, a primeira instituição a 
oferecer um curso de \emph{design} em nível superior no 
território brasileiro \cite{fuchs}. 

\begin{figure}[htb]
	\centering
	\includegraphics[width=8cm]{wollner.jpg}
	\caption{
Composição com Triângulo Proporcional, Alexandre Wollner, 1953,
Museu de Arte Contemporânea da USP.}
	\label{fig:wollner}
\end{figure}

A influência também se manifestou através de figuras 
individuais, sendo Alexandre Wollner, 
pioneiro do \emph{design} gráfico no Brasil,
o caso mais emblemático
(Fig.~\ref{fig:wollner}); 
após estudar em Ulm a convite de Max Bill, Wollner 
retornou ao Brasil e fundou o escritório \emph{Forminform}, 
consolidando o uso do \emph{grid} e do rigor estrutural 
em marcas icônicas como Elevadores Atlas e Banco Itaú
\cite{mizanzuk}.

Outro nome brasileiro a ser citado é o de Luiz Sacilotto,
que embora não tenha estudado diretamente em Ulm,
foi fortemente influenciado por Max Bill e pelo Concretismo de
Theo van Doesburg, que é relacionado aos 
ideais da Bauhaus \cite{fuchs}.
Sacilotto foi um dos pioneiros da 
\emph{Arte Concreta} no Brasil e uma de 
suas obras é tema de uma de nossas
sequências didáticas. 	
	
Fora da Europa e do Brasil, outro nome 
importante no contexto da composições 
geométricas de caráter mais matemático 
é o de Crockett Johnson 
(pseudônimo de David Johnson Leisk).
Seu trabalho representa uma interseção única entre a 
arte e a matemática, fruto de uma ``odisseia'' 
iniciada nos últimos dez anos de sua vida 
(1965--1975) \cite{stroud}. 
Johnson dedicou-se a criar pinturas abstratas 
que documentam marcos matemáticos históricos e 
investigações próprias.
Johnson executou suas pinturas em um estilo de bordas 
rígidas (\emph{hard edge}) e massa plana, 
utilizando cores que focavam em intensidade ou 
contraste para destacar o sentido dos teoremas. 
Seu objetivo era retratar a beleza oculta da matemática 
sem o uso de símbolos, equações ou palavras, 
permitindo que as formas geométricas falassem por si mesmas
(Fig.~\ref{fig:reciproc}, acima).
	
% \begin{figure}
% \begin{center}
% \includegraphics[width=8cm]{./media/image13.png}
% \end{center}
% \caption{Proof of the Pythagorean Theorem (Euclid) --- Crockett Johnson,
% 1965,  National Museum of American History.}
% \label{fig:Pythagorean}
% \end{figure}

\section{Nossa Pesquisa}

A origem de nossa pesquisa foi um trabalho
final, unindo arte e geometria, que o segundo 
autor (Vinícius) propôs aos alunos da disciplina 
Geometria Euclidiana Plana no início de 2024. Foi 
neste momento que conhecemos os trabalhos 
de Crockett Johnson e Max Bill. Após um pouco 
de experimentação, ele percebeu que o GeoGebra
tinha todos os recursos necessários para 
reproduzir boa parte das obras desses autores
(Fig.~\ref{fig:reciproc}, abaixo),
e perguntou a primeira autora (Maiara) se ela
gostaria de trabalhar com esse tema em 
sua dissertação de mestrado. Ela aceitou e 
iniciou a pesquisa que resumiremos a seguir
(continuaremos a usar a primeira pessoa do 
plural no texto, mas a pesquisa foi conduzida
de forma bastante autônoma pela primeira autora).

\begin{figure}[htb!]
\begin{center}
\includegraphics[width=6cm]{reciprocation.jpg} 
\includegraphics[width=6cm]{CJ1.png} 
\end{center}
\caption{\emph{Reciprocation}, de Crockett Johnson, acima e
nossa reprodução no GeoGebra abaixo. A pintura ilustra o 
Teorema de Pappus.}
\label{fig:reciproc}
\end{figure}

A fim de responder nossa pergunta 
inicial, definimos que a nossa pesquisa 
teria caráter \emph{qualitativo}, 
 o qual permite a observação, compreensão e 
 análise dos sujeitos interagindo com os 
 conhecimentos pretendidos, em busca de entender 
 os processos que os levam aos resultados e não 
 apenas focar nos resultados finais 
 \cite{bogdan}. Quanto aos objetivos do estudo, classificamos a 
pesquisa como  \emph{exploratória}, já que concordamos 
com Lunetta e Guerra \cite{lunetta} sobre esse %p.2 
tipo de pesquisa ser utilizado quando se 
pretende aprofundar os conhecimentos sobre o 
conteúdo pesquisado, permitindo desenvolver 
hipóteses e maior especificidade no que se conhece 
sobre o tema.

Após um período de estudo e planejamento,
formulamos os seguintes objetivos específicos:
\begin{itemize}
\item Desenvolver sequências didáticas que utilizem composições geométricas para o ensino de conteúdos matemáticos;
\item Apresentar o desenvolvimento de uma oficina sobre Composições Geométricas com o auxílio do GeoGebra;
\item Investigar a percepção dos estudantes sobre as sequências didáticas e sobre a oficina;
\item Entender a viabilidade do ensino de conteúdos matemáticos a partir da interdisciplinaridade com a arte.
\end{itemize}


O local escolhido para a realização da pesquisa foi 
um Colégio Estadual situado na zona urbana da cidade de 
Seabra-Bahia, onde a pesquisadora exerce 
a docência. Ele atende majoritariamente estudantes do 
Ensino Médio, Ensino Médio Técnico e Educação de Jovens e 
Adultos (EJA), funciona nos turnos matutino, vespertino, 
noturno e integral e possui um polo da Universidade 
Aberta do Brasil (UAB), que oferece cursos de graduação e 
pós-graduação. Solicitamos à coordenação o uso do 
laboratório de informática UAB, localizado no 
interior da escola estadual, pois optamos em 
desenvolver as atividades em computadores, 
ao invés de celulares ou \emph{tablets}. 
Fizemos essa opção metodológica por entendermos que 
os computadores proporcionariam melhor visualização 
das construções e seus detalhes. No momento da visita 
inicial ao laboratório, havia 12 computadores com 
bom funcionamento e decidimos limitar o número 
de participantes à quantidade de máquinas. A 
direção escolar e a coordenação da UAB Seabra-BA 
assinaram as respectivas declarações de anuência, 
concedendo a permissão para a realização da pesquisa. 

Sabendo com precisão o local de aplicação que a 
pesquisa seria desenvolvida --- o laboratório de informática ---
foi possível definir os convidados para a realização da 
oficina. Decidimos convidar estudantes que 
tivessem disponibilidade para estar na escola no 
turno oposto ao que estudavam e assim surgiu a 
oportunidade de convidar os participantes do 
Clube de Ciências da escola (MegamenteCES), os 
quais já frequentavam a escola no turno oposto para 
reuniões. 
%Desse grupo, 
%efetivamente participaram quatro estudantes. No momento 
%da pesquisa, um estudante frequentava a primeira 
%série no turno integral e três frequentavam a terceira 
%série do Ensino Médio regular, no turno matutino, 
%com idades entre 15 e 18 anos.

Como o quantitativo do público esperado não havia 
sido atingido estendemos 
o convite para
estudantes de uma turma de quarto 
semestre de Licenciatura em Matemática à 
distância, pertencente à UAB --- Seabra,
e uma manifestou interesse em participar. 
No momento do convite, acreditávamos que ter 
participantes do Ensino Médio e do Ensino Superior 
traria visões complementares e nos ajudaria na 
melhor compreensão sobre os efeitos da atividade 
interdisciplinar no ensino da matemática. 
Posteriormente, um estudante que cursava a 
terceira série, mas não fazia parte do clube de ciências, 
se mostrou interessado em compor o grupo.
Dessa forma, o grupo final foi composto por sete estudantes, 
sendo seis do Ensino Médio e um do Ensino Superior.
Todos os participantes assinaram o Termo de Consentimento Livre e Esclarecido
(TCLE) para participar da pesquisa.

Quanto ao conteúdo das Oficinas, optamos por nos
limitar ao estudo das Transformações Isométricas
e ao Teorema de Pitágoras, por serem conteúdos
do Ensino Médio que podem ser explorados
através de composições geométricas.
Transformações Isométricas são transformações que 
conservam a distância (\cite{lima1996isometrias}). 
O ensino de Isometrias está previsto na 
BNCC do Ensino Médio, inclusive tendo como sugestão a 
sua utilização para análise de obras de arte:
\begin{quote}
    (EM13MAT105) Utilizar as noções de transformações 
		isométricas (translação, reflexão, rotação e 
		composições destas) e transformações 
		homotéticas para analisar diferentes 
		produções humanas como construções civis, 
		obras de arte, entre outras \cite{bncc}.
\end{quote}


Com os conteúdos matemáticos definidos, 
realizamos a estruturação das sequências didáticas. 
Elas são iniciadas com o conceito matemático que se 
pretende trabalhar, para que o estudante comece a 
refletir sobre a matemática envolvida. Nesse momento 
pode haver uma conversa sobre o tema entre docente 
e discentes. Em seguida, há sessões da sequência 
com as construções propostas. Cada sessão conta 
com o passo a passo da construção que se pretende 
realizar no GeoGebra e, prosseguindo, são feitos 
questionamentos para incentivar que os participantes 
reflitam sobre o que estão elaborando e sintetizem 
seus aprendizados. Ao final de cada sessão de construção, 
há uma sessão chamada ``para finalizar'' onde são 
dadas instruções sobre como salvar, exportar 
ou estilizar a produção.

Sabendo que os estudantes podem não ter familiaridade
com o GeoGebra, organizamos uma sessão inicial 
para que os participantes conheçam o aplicativo 
com o auxílio da docente responsável. Elaboramos alguns 
vídeos para apoiar os momentos de prática, sobre: 
informações iniciais relativas ao GeoGebra; 
como exportar imagens do GeoGebra para o computador; 
como modificar as cores no GeoGebra. Ademais, 
elaboramos e disponibilizamos vídeos sobre os artistas 
dos quais pretendemos realizar construções das obras 
de arte, que são: Max Bill, Luiz Sacilotto e Crockett 
Johnson. O \emph{link} de cada vídeo está disponível ao 
longo das sequências didáticas e também na página 
\emph{online} organizada para uso durante as oficinas
\cite{padlet}.

Foram elaboradas três sequências didáticas com 
composições geométricas para serem desenvolvidas 
no GeoGebra. Elas buscam despertar nos 
participantes o caráter investigativo, 
contando com o auxílio de um docente que fará 
a mediação das atividades, mas permitindo 
que os discentes desenvolvam as atividades 
seguindo as orientações escritas nas sequências didáticas. 

As duas primeiras sequências se complementam, 
tratando do conteúdo matemático isometrias 
no plano: a primeira sequência trabalhando 
com produções que envolvem reflexão e rotação, e 
a segunda sequência focando no estudo de translações. 
Para estas sequências didáticas, exploramos tanto 
construções mais livres em que estudaríamos os 
conceitos a partir das produções de cada estudante e 
construções das obras de arte de Max Bill e Luiz Sacilotto. 
A terceira sequência explora o teorema de 
Pitágoras e a média geométrica a partir de uma 
obra do artista Crockett Johnson.

\section{Execução e Resultados}

As oficinas ocorreram em três encontros presenciais
 de duas horas cada, no laboratório de informática, 
 com datas escolhidas em diálogo com os participantes 
 para minimizar ausências. Sete alunos participaram 
 do primeiro encontro (identificados como A, B, C, D, E, F e G); 
 seis no segundo e cinco no terceiro.
 

\begin{figure}[htb!]
    \centering
    \includegraphics[width=8cm]{Figuras/padlet.png}
    \caption{Mural interativo elaborado para as oficinas \cite{padlet}.}
    \label{fig:padlet}
\end{figure}

No \textbf{primeiro encontro}, após 
apresentar o mural colaborativo (Fig.~\ref{fig:padlet}),
contendo todos os materiais de apoio, 
e o GeoGebra, os alunos 
iniciaram a sequência didática que 
consiste das seguinte construções:
\begin{enumerate} 
\item Reflexão utilizando comando do GeoGebra; 
\item Rotação utilizando comando do GeoGebra;
\item \emph{Composition Géometrique}, 1990, Max Bill
\item \emph{Concreção 8745} de Luiz Sacilotto.
\end{enumerate}

A primeira construção consistia 
em efetuar reflexões de 
um polígono em relação aos eixos e à origem. 
Ao comparar 
coordenadas, perceberam as mudanças provocadas 
pelas reflexões e surpreenderam-se ao mover o 
polígono original e verificar que as imagens 
refletidas o acompanhavam. O estudante 
F comentou: ``acompanharam a posição do polígono 
inicial, porém permaneceram invertidos''.

Em seguida, exploraram rotações da obra 
referencial em $45^\circ$, $90^\circ$, $180^\circ$ e $270^\circ$ 
em torno da origem, construindo círculos 
para destacar a preservação de distâncias. O 
estudante B observou que ``o círculo passa sempre 
nos mesmos vértices'' e a estudante E notou 
que os círculos passavam por todas as 
imagens rotacionadas, mesmo ao mover a 
figura inicial. Por fim, utilizaram controle 
deslizante para visualizar rotações contínuas 
em torno da origem e de um vértice.

\begin{figure}[htb!]
    \centering
    \includegraphics[width=7cm]{Figuras/maxbill.jpg}
    \caption{\emph{Composition Géométrique}, Max Bill (1990).}
    \label{fig:maxbill}
\end{figure}


O \textbf{segundo encontro} focou na reprodução 
de obras de arte. Após assistir a vídeo sobre Max Bill, 
os alunos analisaram \emph{Composition Géométrique} (1990)
(Fig.~\ref{fig:maxbill}). 
Destacam-se as observações do estudante F ---
 ``diversos triângulos unidos em diferentes sentidos'' 
 e formação de um quadrilátero central (losango) ---
  e do estudante B --- 
	``dividida em 4, cada imagem em ângulo diferente, 
	dá a impressão que está rodando em espiral''. 
	Fragmentos das respostas reforçam a percepção de 
	simetria, repetição de cores e padrões 
	rotacionais de $90^\circ$.

Os alunos reproduziram um quarto da obra e 
completaram o restante usando rotações e reflexões. 
Três participantes (A, B e F) demonstraram maior autonomia; 
os demais necessitaram de mais apoio, especialmente no 
manuseio preciso do mouse e na extração de cores RGB.

\begin{figure}[htb!]
    \centering
    \includegraphics[width=7cm]{Figuras/Concrec.jpg}
    \caption{\emph{Concreção 8745}, Luiz Sacilotto.}
    \label{fig:concrecao}
\end{figure}

Ao final, iniciou-se a análise da obra 
\emph{Concreção 8745} de Luiz Sacilotto (Fig.~\ref{fig:concrecao}), 
com percepções sobre simetria, harmonia 
cromática e possibilidade de obter 
formas inferiores por reflexão.

O \textbf{terceiro encontro} foi dedicado à 
conclusão da reprodução de \emph{Concreção 8745}. 
Os alunos construíram uma parte triangular (oitavo da obra), 
coloriram-na e completaram o quadrado central azul-amarelo 
usando reflexões. Para finalizar a obra inteira, 
empregaram reflexões em relação a segmentos internos. 
Os estudantes A, B e F atuaram com autonomia; C, D e E 
requereram maior intervenção da docente. O estudante 
A, tendo já concluído, criou composição original inspirada 
nas oficinas (Fig.~\ref{fig:composicaoA}).

\begin{figure}[htb!]
    \centering
    \includegraphics[width=7cm]{Figuras/compA.png}
    \caption{Composição geométrica de autoria do estudante A.}
    \label{fig:composicaoA}
\end{figure}

%Ao término, os participantes responderam questionário \emph{online} (Google Forms) com 20 perguntas sobre perfil tecnológico, percepções acerca do uso de ferramentas digitais, relação matemática-arte e experiência com o GeoGebra nas oficinas.





O questionário \emph{online} (Google Forms), 
intitulado ``Questionário sobre as oficinas: composições 
geométricas no GeoGebra'', foi respondido pelos 
cinco participantes presentes na última oficina. 
Composto por 20 perguntas de múltipla escolha e 
discursivas, buscou investigar perfil tecnológico, 
percepções sobre tecnologias digitais em aulas de 
matemática, relação matemática-arte, experiência com 
o GeoGebra, dificuldades enfrentadas e sugestões 
sobre as oficinas.

Todos os respondentes declararam ter acesso à 
internet em casa e na escola, principalmente 
via \emph{smartphone} (100\%), \emph{notebook} 
(80\%), \emph{tablet} (40\%) e computador (20\%). 
Quanto à experiência com computadores, todos 
já possuíam alguma, sendo 60\% em nível básico e 
40\% intermediário.

Os participantes consideraram, de forma unânime, 
ser importante o uso de tecnologias digitais em 
aulas de matemática. Destacam-se justificativas como: 
``tecnologias digitais têm muitas ferramentas 
que tem forte ligação com a matemática''; 
``nos ajuda a entender a matemática de maneira 
dinâmica e didática''; ``enriquece as aulas''; 
``acompanha a evolução que acontece na sociedade atual''. 
Tais percepções reforçam a contribuição das 
ferramentas digitais para o engajamento e 
construção significativa de conceitos matemáticos, 
conforme aponta  Cox em \cite{cox}.

Apenas uma participante (licencianda em Matemática) 
já havia utilizado o GeoGebra anteriormente, 
mas todos o consideraram uma boa ferramenta 
para aprender matemática, destacando: 
``diversidade de funções''; 
``aproximação prática com a teoria''; 
``facilita o aprendizado e demonstra melhor as 
figuras visualmente''; ``oportunidade de conhecer 
várias formas geométricas através da arte''. 
Como aspectos mais interessantes, citaram o 
aprendizado prático e a abordagem via arte.

O grupo concordou que relacionar matemática 
e arte é uma boa estratégia pedagógica, pois 
permite explorar criatividade, estudar de 
forma dinâmica, compreender conexões 
interdisciplinares e reproduzir processos 
artísticos. Essa abordagem evidencia a 
presença da matemática em contextos culturais e 
sua potencialidade interdisciplinar, 
corroborando Mendes ao afirmar que 
atrelar geometria a manifestações artísticas 
torna o aprendizado mais atraente e significativo, 
extensão válida também para adolescentes e adultos \cite{Mendes2012}.

Quanto às dificuldades, 60\% 
relataram problemas devido à falta de familiaridade 
prévia com o GeoGebra ou baixa frequência 
no uso de computadores, mas enfatizaram a 
importância das orientações da docente para 
superar os obstáculos.

Todos aprovaram o formato das oficinas 
(orientações presenciais, sequência didática 
impressa e uso de computadores), destacando trechos 
como: ``é como se houvesse um guia''; 
``metodologia muito boa''; 
``gostei muito desse formato''; 
``tivemos autonomia''.

Nos comentários finais, todos foram positivos, 
elogiando a experiência e sugerindo a inclusão 
de momento para criações autorais no GeoGebra. 
Trechos relevantes: ``quero levar como conhecimento 
para aprimorar meus estudos''; ``ganhei experiência 
em novas plataformas digitais''; ``aprendi 
mais sobre a matemática e a arte''; ``excelente 
oportunidade de aprendizado sobre o aplicativo e 
suas funcionalidades''.

\section{Considerações Finais}

Infelizmente, não foi possível aplicar
todas as sequências didáticas planejadas,
o que sugere que uma Oficina como esta
necessita de mais tempo para ser
desenvolvida. O número reduzido de participantes
também limitou a abrangência dos resultados,
mas, mesmo assim, foi possível perceber
que a proposta foi bem recebida pelos estudantes
e que eles conseguiram compreender os conceitos
matemáticos trabalhados.
Acreditamos que a interdisciplinaridade entre
matemática e arte, aliada ao uso de tecnologias
digitais, pode ser uma estratégia eficaz	
	 para o ensino de conteúdos matemáticos,
como as transformações isométricas. 
Esperamos que este trabalho
incentive outros educadores a explorar
essas conexões em suas práticas pedagógicas.



\bibliography{didatica}

\vfill


\pagebreak 

% Mini bios 
% Seja informal e divertido
% Prefira fotos com fundo branco
\begin{wrapfigure}{L}{1.7cm}
	\centering
	\includegraphics[width=2cm]{maiara.jpg}
\end{wrapfigure}\noindent
Maiara Santos é soteropolitana e reside na cidade 
de Seabra-BA. É especialista em Educação Matemática 
e atua principalmente em turmas de ensino médio.  
Além de ser apaixonada por Dança e Literatura, 
é encantada pela versatilidade da Geometria e com o 
seu potencial artístico. Por isso, sempre se 
utiliza de alguma conexão entre Matemática e 
Arte em suas aulas. 

\vspace{1cm}
\begin{wrapfigure}{L}{1.7cm}
\vspace{-10pt}
  \includegraphics[width=2cm]{Vinicius.jpg}
\end{wrapfigure}\noindent
Vinícius Mello nasceu em Salvador e obteve seu doutorado
em Computação Gráfica no IMPA. Ensina matemática 
na UFBA e fica alegre sempre que pode usar o GeoGebra
em suas aulas. Gosta de matemática, música e programação 
em exata proporção, podendo ser encontrado a
(quase) qualquer momento fazendo ao menos uma dessas coisas.

\end{document}
