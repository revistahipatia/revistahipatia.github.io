\documentclass[onecolumn]{hipatia}
\usepackage{blindtext}
\newcommand{\superau}{\textsuperscript{\underline{a}}~}
\title{Arte e Matemática}
\subtitle{Epístola}
\author{}
\begin{document}
\setcounter{page}{\epistolapage}
\maketitle
\leftskip=2.5cm
\rightskip=2.5cm

\vspace{1cm}

\noindent Caro Leitor,
\vspace{1cm}

Matemática, Arte e Ciência são campos do conhecimento 
humano que, embora distintos em seus métodos e objetivos, 
frequentemente se entrelaçam de maneiras surpreendentes e 
inspiradoras. 
Nesta edição da \textsc{Revista de Matemática Hipátia},
exploramos a interseção entre esse domínios,
destacando como eles se complementam e se enriquecem
mutuamente.

Na seção \textsc{História}, a professora Ana Lucia
apresenta a primeira parte de uma série de 
dois artigos em que ela, mais do que traçar a evolução
da fotografia e sua conexão com a ciência e a matemática,
nos propõe uma reflexão sobre o papel das 
imagens, sejam analógicas ou digitais, como 
mediadoras entre o homem e a realidade, 
com todas as consequências sociais e políticas
que isso acarreta. Ana Lucia, além de matemática
de formação, está concluindo seu mestrado na
Escola de Belas Artes da UFBA, o que lhe confere
uma perspectiva única para abordar esse tema.

A relação entre Arte e Matemática é aprofundada
na seção \textsc{Didática}, onde
os autores (Maiara Santos e eu) exploram 
como a arte pode ser uma ferramenta poderosa
para o ensino da matemática. Após uma introdução aos principais
elementos e princípios das composições geométricas,
o artigo relata uma oficina realizada com estudantes
de Seabra--BA, utilizando o \emph{software} GeoGebra para
reproduzir obras de arte concretistas e explorar
transformações isométricas. Essa é primeira vez que
a \textsc{Revista de Matemática Hipátia} publica
um artigo de pesquisa em ensino de matemática, 
conduzido no escopo do PROFMAT/UFBA (Mestrado Profissional
em Matemática em Rede Nacional/UFBA),
reforçando nosso compromisso
com a educação matemática de qualidade.

Na seção \textsc{Técnica}, o autor Armando Barbosa
apresenta um método para resolver equações polinomiais
de grau 3, utilizando as fórmulas de Tartaglia--Cardano.
Não custa lembrar que o título do livro em que
Cardano publicou essas fórmulas (gerando uma 
polêmica célebre com Tartaglia) é \emph{Ars Magna},
ou seja, \emph{A Grande Arte}, o que evidencia
a fronteira tênue entre a matemática e a arte,
ao menos no início da Idade Moderna.

Na tradicional seção \textsc{Problema}, temos 
a resolução dos problemas da edição anterior
e mais alguns novos desafios para os leitores, 
que certamente requererão engenho e \dots arte.

Trazemos uma 
retrospectiva das atividades realizadas 
pelo Departamento de Matemática (DMAT) da UFBA em 2025
na seção \textsc{Simpósio}. 
A seção destaca eventos de formação de professores, 
congressos científicos como o XVIII ENAMA e o X EPGMAT,
encontros como o EMAT e o Seminário de IC,  
projetos de extensão como o PECMat (egressos), Café Cultural e o 
coletivo Ondjango Asili (jogos africanos), 
além das cerimônias de premiação de olimpíadas de 
matemática como a OBMEP e a OMEBA.

Para encerrar com chave de ouro essa edição 
recheada de arte e matemática, apresentamos a seção 
\textsc{Fotografia},  
onde a professora Cristina Lizana 
apresenta algumas de suas belas fotografias,
as quais manifestam um olhar artístico permeado
por sensibilidade matemática.

Na capa desta edição, temos uma composição 
criada pelo professor Nicola Sambonet, na qual 
uma carta do matemático Euler, em que
ele descreve o funcionamento de um conjunto 
de lentes, está 
sobreposta a uma fotografia
(de autoria de
Kenny Louie)
de uma 
câmera Hasselblad. São muitas camadas
de significado: um matemático quase
cego descrevendo
a ciência das lentes, 
muito antes da invenção
de um instrumento composto de lentes,
 a câmera fotográfica,
que é capaz de registrar imagens,
como a imagem de uma câmera! 


\vspace{1cm}

\hfill Salvador, 29 de dezembro de 2025.

\hfill O Editor
\end{document}
