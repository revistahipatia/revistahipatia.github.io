\documentclass{hipatia}
\usepackage{lipsum}
%Use \DeclareMathOperator para definir novos
% operadores para o modo matemático
\DeclareMathOperator{\sen}{sen}

%Evite numerar teoremas
%Prefira nomeá-los
%Use os ambientes abaixo
\newtheorem*{theorem*}{Teorema}
\newtheorem*{lemma*}{Lema}

% Evite títulos muito longos
\title{Ciências e Fotografia:\\ 500 anos de História (I)}
% Se for necessário diminuir a fonte do título 
% para caber no quadro, use
% \title{ \fontsize{28}{28}\selectfont Uma Nova Demonstração do\\  \fontsize{28}{28}\selectfont Teorema de Pitágoras}

% O Subtítulo é o nome da seção da revista
% Deve ser uma palavra de origem grega
\subtitle{História}
\author{Ana Lucia Pinheiro Lima}
% A data não é necessária
%\date{October 2023}
% Não se preocupe com a numeração
% das páginas ou com o número da edição
\usepackage{hanging}

\begin{document}
\setcounter{page}{\historiapage}
\maketitle

\section{Introdução}

A humanidade, depois de aproximadamente duzentos mil 
anos de história, chega ao vigésimo quinto ano 
do século XXI imersa em uma organização social 
caracterizada por amplas e profundas transformações.

O tempo da nossa contemporaneidade é urgente e 
ritmado pelos mais recentes desenvolvimentos 
tecnológicos. Nas últimas décadas, formou-se 
uma complexa e abrangente rede de invenções 
que abrem possibilidades antes inimagináveis 
na história da humanidade e que se auto desafiam 
numa corrida pelo posto de mais avançada e poderosa. 
Essa tecnologia, que hoje se materializa em 
nossas mãos na forma de um telefone celular, 
tem suas raízes no século XVI, quando começou 
a Revolução Científica que, por sua vez, 
herdou os frutos das investigações acumuladas 
ao longo dos séculos anteriores. É o 
conhecimento científico construído pela mente 
humana que sustenta essa rede.

Paradoxalmente, a superfície lisa e fria 
das telas funciona como uma sedutora barreira, 
que nos separa e distancia de toda essa 
complexidade. Diante das telas, nossos olhos 
são inundados pela luz. As imagens, mosaicos de 
pixels, apresentam-se como realidade, uma 
realidade sem contradições e, portanto, 
sem questionamentos. Somos convidados a seguir 
o fluxo imposto por escolhas feitas em outras 
camadas, inacessíveis ao usuário hipnotizado 
pelo espelho negro.

Mas, se o questionamento é caminho para o 
raciocínio crítico e este é terreno fértil 
para o desenvolvimento de novos conhecimentos, 
é válido perguntarmos sobre como essa espécie 
de inércia intelectual gerada pela submissão 
às imagens digitais, que são o código, a 
linguagem da vida na sociedade informatizada, 
se reflete na investigação da própria 
realidade e, consequentemente, no mundo do 
ensino e da pesquisa científica.

As possíveis abordagens para investigar essa 
questão são tão abrangentes quanto as 
circunstâncias em que ela se insere. O 
conjunto de aspectos que contorna a nossa 
relação com a tecnologia perpassa os 
principais aspectos da construção do 
conhecimento: Ciências Exatas, Humanas, 
Sociais, da Saúde, Biológicas, da Terra, 
Arte, Religião; toda  forma de investigar e 
explicar a experiência humana foi 
atingida pelo mundo digital.

Antes de analisarmos os reflexos desses 
novos ``padrões de comportamento'', dessa 
nova estrutura social modelada pelas 
novas tecnologias, é fundamental encararmos 
essa tecnologia, é urgente espiarmos por 
trás das telas. Como no mito da caverna 
de Platão, para entendermos as circunstâncias 
em que estamos inseridos, é preciso olhar 
de onde vem a luz.

A proposta desse texto, que será publicado 
em duas edições da Revista de Matemática Hipátia, 
é construir uma linha do tempo, dos últimos 
500 anos, apresentando paralelos entre 
algumas das principais descobertas/invenções 
das ciências e a evolução do registro de 
imagens criadas com a luz. Da investigação 
a respeito da natureza da luz, até a criação 
de imagens pelas inteligências artificiais, 
vamos revisitar alguns dos principais 
resultados científicos alcançados nesse 
período. Nessa primeira parte do texto, 
falaremos dos acontecimentos até o final 
do século XIX. Na próxima edição, voltaremos 
ao início do século XX procurando os 
primeiros movimentos que culminaram na 
revolução digital, até chegarmos às novas 
(novas?) imagens geradas por inteligência 
artificial.

Recuando um pouco no tempo e estabelecendo 
uma pequena distância do \emph{tsunami} de 
imagens em que estamos mergulhados, poderemos 
refletir criticamente sobre as estruturas 
que moldam e sustentam a sociedade contemporânea 
e, consequentemente, poderemos ter a chance 
de sairmos da indiferença e do tédio diante 
dessas pseudo-imagens-verdades.


\section{A revolução científica e a pesquisa sobre a luz}


No início do século XVI, as novas rotas marítimas 
geraram uma primeira onda de globalização, 
desencadearam a expansão do comércio e o 
confronto de civilizações; promoveram conquistas 
e infortúnios. Influenciaram no fim do feudalismo 
na Europa e promoveram uma era de exploração, 
com o surgimento do colonialismo. Foi também 
um tempo de inovação nas artes e de questionamentos 
sobre a autoridade religiosa, confrontada pela 
ideia de que o ser humano podia dar sentido à 
própria vida.

Os valores humanistas tiveram grande influência 
nas investigações sobre a realidade. A partir 
desse momento, a observação dos fenômenos e 
o posterior questionamento sobre eles, 
privilegiando o rigor do raciocínio, seriam 
a base da pesquisa científica.

No livro ``De Revolutionibus Orbium Coelestium'' 
(Sobre as Revoluções das Esferas Celestiais [4]), 
publicado em 1543, Nicolau Copérnico (Polônia, 1473) 
apresentou um modelo para o universo onde, 
pela primeira vez, a Terra não ocupava o seu 
centro. As ideias de Copérnico divergiam do 
que era defendido pela igreja e também de 
clássicas teorias gregas. 

No século seguinte, Galileo Galilei (Itália, 1564), 
utilizando um telescópio, invenção aprimorada 
por ele a partir de uma primeira versão holandesa, 
para observar o movimento de luas de Júpiter, 
comprovou que as teorias geocêntricas, centradas 
na Terra, para o universo não poderiam estar 
corretas, fortalecendo as ideias de Copérnico.

Em 1632, Galileo publica ``Dialogo sopra i due 
massimi sistemi del mondo'' (Diálogo sobre os 
dois principais sistemas mundiais [11]), 
comparando os modelos geocêntrico e heliocêntrico, 
centrado no sol.  Por questionar a influência 
dos dogmas religiosos dominantes à época sobre 
o conhecimento científico, Galileo foi julgado 
e condenado por heresia. 

A pesquisa realizada por Galileo é um marco da 
história das ciências,  pois apresenta um 
equilíbrio entre experimentos e teorias, 
que é a base do hoje chamado 
\textit{método científico}:
\begin{center}
    observações $\rightarrow$ perguntas $\rightarrow$ hipóteses  $\rightarrow $ experimentações  $\rightarrow$ conclusões $\rightarrow$  observações $\rightarrow$  perguntas  …
\end{center}

	A curiosidade humana sempre se interessou 
	pelos fenômenos relacionados à luz. O primeiro 
	registro sobre a chamada \textit{câmara escura}, 
	uma caixa ou sala fechada, com uma única pequena 
	abertura em uma das paredes por onde passa a luz, 
	projetando-se na parede oposta, encontra-se em 
	um texto do filósofo chinês Mozi do século 
	V a.c. [16]. Tanto Mozi, como também o matemático 
	grego Euclides, um século mais tarde, perceberam 
	que a luz move-se em linha reta. Assim, a 
	imagem projetada na parede de uma câmara escura 
	era refletida de cima para baixo e da esquerda 
	para a direita.
    
	Foi no século XI que o matemático e físico 
	persa Alhazen (Iraque, c. 965) escreveu  
	``Kitab al-Manazir''  \, (Livro de óptica [27]), 
	o primeiro tratado sobre os fenômenos envolvendo 
	a luz. No texto, ele apresenta semelhanças de 
	como a luz afeta o olho humano com o que 
	acontece quando a luz penetra na câmara escura. 
	Também argumenta que a visão acontece no 
	cérebro e, portanto, seria influenciada 
	pela experiência pessoal. Alhazen foi um 
	precursor do método científico. Seu trabalho 
	abrangeu, além de matemática e física, 
	filosofia, teologia e medicina, e seus 
	escritos foram referência para muitas 
	pesquisas a partir do século XVI.


    
\begin{figure}[htb!]
\centering
\includegraphics[width=8cm]{MICHELANGELO-1545.jpg}
\caption{Michelangelo, 1545}   
\end{figure}

Durante a Renascença,  além de ser usada na 
observação de eclipses e outros fenômenos 
celestes, a câmara escura foi usada por 
artistas como instrumento auxiliar para 
desenho e pintura. Leonardo da Vinci 
(Itália, 1452) foi um desses artistas. 
Em seus cadernos [5], aparecem centenas de 
desenhos de câmaras escuras. Na época, 
espelhos e lentes passaram a compor as 
câmaras, de modo a melhorar a qualidade 
da imagem projetada. 
    
O matemático e astrônomo Johannes Kepler 
(Alemanha, 1571), contemporâneo de Galileo, 
também se interessou em investigar fenômenos 
astronômicos. Kepler construiu, dentre outras, 
a teoria matemática que garante a forma 
elíptica da trajetória dos planetas em torno 
do sol. Suas observações sobre os corpos celestes, 
utilizando a câmara escura, o levaram a também 
dedicar seus estudos para compreender a 
óptica. ``Astronomiae pars optica'' 
(A parte óptica da astronomia [17]) foi 
publicado em 1604. O trabalho de Kepler 
estabelece as leis matemáticas da chamada 
\textit{Geometria Óptica}, 
demonstrando as leis geométricas que 
descrevem o fenômeno da produção de 
imagens feitas pela luz no interior de 
uma câmara escura e também os efeitos da 
luz no interior do olho humano.  Kepler, 
a partir de resultados da anatomia humana 
conhecidos na época, explicou como as imagens 
se formavam na retina do olho e não no 
cristalino, como se acreditava até então.

Na pesquisa de Kepler, as leis matemáticas 
da astronomia e da óptica se encontram. 


Também é importante destacar que o trabalho 
de Kepler delimita quais aspectos do 
fenômeno da visão caberia à óptica e a 
geometria, isto é, o que acontecia no olho. 
O caminho que a imagem formada na retina 
seguia a partir dali era outro fenômeno. 
Era um prenúncio de um tempo de especialização 
das pesquisas científicas.

Em seus cálculos, Galileo e Kepler usam 
argumentos que mais tarde seriam fundamentados 
matematicamente na teoria do cálculo infinitesimal 
elaborada por Isaac Newton e Gottfried Leibniz, 
no século seguinte, de maneira independente.

Na segunda metade do século XVII, o cientista 
Isaac Newton (Inglaterra, 1642), considerado 
por muitos como o maior de todos os cientistas, 
apresentou o conjunto de leis que regula o 
movimento dos corpos e a lei da gravitação 
universal, generalizando os resultados de 
Galileo e Kepler. Seu livro ``Philosophiae Naturalis 
Principia Mathematica'' 
(Princípios matemáticos da filosofia natural [22]) 
foi publicado em 1687. As pesquisas feitas por 
Newton também envolviam a luz. Em ``Optics'' 
(Ótica [23]), publicado em 1704, ele apresentou 
a teoria sobre a \textit{dispersão da luz}: a 
separação da luz branca (luz do sol) em um 
espectro de sete cores, ao passar por um 
meio transparente. 

O trabalho de Newton foi profundamente 
influenciado por René Descartes, matemático, 
físico e filósofo francês (1596), autor 
da célebre frase ``Penso, logo existo.'',
e do tratado ``Discours  de la méthode''  
(Discurso do método [6]), publicado em 1637, 
onde ele apresenta o \textit{método dedutivo} 
para investigação científica baseado em quatro 
etapas: evidência, análise, síntese, verificação. 
Criador da geometria analítica, que usa a 
simbologia algébrica para descrever objetos 
geométricos, Descartes também investigou 
fenômenos envolvendo a luz. Em 
``La Dioptrique'' $\,$ (A dióptrica [7]), 
de 1637, aplicou o método dedutivo para 
explicar matematicamente o fenômeno da 
\textit{refração da luz}: quando a luz 
muda de direção ao mudar de um meio de 
propagação para outro. Os trabalhos de 
Descartes e Newton completam a teoria 
óptica que explica o fenômeno do arco-íris.

O matemático e físico Christiaan Huygens 
(Holanda, 1629) também publicou importante 
trabalho sobre a natureza da luz e suas 
propriedades. Em seu livro ``Traité de la lumière''
(Tratado sobre a luz [15], 1690), apresenta uma 
teoria ondulatória para explicar a natureza da 
luz. Huygens é considerado um dos primeiros 
cientista a usar o rigor matemático para 
explicar um fenômeno não observável [8].  
Com o avanço do uso dos símbolos algébricos 
como ferramenta para tratar assuntos complexos, 
a pesquisa em matemática que até então se 
inspirava em fenômenos da natureza e 
dedicava esforços a explicá-los, 
passou a ser capaz de investigar 
problemas abstratos.

 
No início do século XVIII,  teorias 
físicas e matemáticas explicando  fenômenos 
envolvendo a luz estavam bem fundamentadas. 
Já sobre a natureza da luz, os estudiosos 
se dividiam em dois grandes grupos: luz 
como onda, como Huygens, e luz como 
partícula, como Newton. 
Foram necessários mais dois séculos de 
pesquisas para que as teorias sobre a natureza dual
da luz, onda e partícula, fossem finalmente estabelecidas.
%Só na segunda 
%metade do século XIX, a teoria que descreve 
%a natureza dual da luz foi estabelecida por 
%James C. Maxwell (Escócia, 1831), cujo 
%trabalho será comentado a seguir.


Embora os efeitos da luz na pele humana 
e nos objetos fossem observados e 
investigados desde a antiguidade, foi a 
partir das pesquisas feitas pelo alquimistas, 
considerados os primeiros químicos, 
durante a idade média, que as propriedades 
que algumas substâncias químicas possuem de 
transmutar (escurecer) sob a ação da luz 
foi investigado com rigor científico.

O primeiro registro da produção da 
substância química \textit{nitrato de prata}, 
conseguida dissolvendo prata em ácido cítrico, 
aparece em escritos do alquimista Geber 
(Irã, c. VIII). Os também alquimistas Angelo 
Sala (Itália, 1576) e Johann Heinrich Schulze 
(Alemanha, 1687) descobriram a propriedade do 
nitrato de prata escurecer sob a ação da luz 
solar [9]. Em 1610, Sala anunciou a descoberta, 
mas como não parecia haver uma aplicação imediata, 
o fenômeno foi deixado de lado, inclusive pelos 
alquimistas. Em 1717, Schulze demonstrou que 
era, de fato, a luz e não o fogo que escurecia 
o nitrato de prata, como alguns acreditavam. 
Ele usou uma ``máscara com texto vazado'' 
para cobrir uma garrafa contendo o nitrato 
de prata e expôs a garrafa à luz do sol. 
Após um tempo de exposição, a porção da 
substância exposta à luz escureceu e, 
retirada a máscara, o texto ficou ``impresso'' 
no vidro. Embora o experimento tenha 
funcionado bem, a impressão durou apenas o 
tempo necessário para que todo o nitrato de 
prata escurecesse. 

A partir daí, iniciou-se a busca por um 
método efetivo de escrita usando a luz. 


\section{A fotografia como produto da revolução industrial}

O desenvolvimento das ciências durante o 
século XVIII foi marcado pela consolidação 
do rigor das metodologias de pesquisa surgidas 
nos séculos anteriores. Na matemática, o uso 
do cálculo infinitesimal deu origem a novas 
áreas de pesquisa, como análise e geometria 
diferencial. A ênfase na razão, as críticas 
às superstições e o questionamento do poder 
absoluto, representados pela monarquia e pela 
igreja, eram marcas do \textit{iluminismo}. 
Os ideais de igualdade e liberdade pregados 
pelo movimento, iniciado na França no 
final do século XVII e que se tornou popular 
durante o século XVIII, ressoaram nas colônias 
europeias nas Américas. Os movimentos de 
independência dos EUA (1776), do Haiti (1804), e 
também a Inconfidência Mineira (1789) e a 
Revolta dos Alfaiates (Bahia, 1798), foram 
movimentos emancipatórios com fortes valores 
iluministas.

O século XVIII também testemunhou o 
desenvolvimento da \textit{revolução industrial}. 
Com a invenção das máquinas, o método de produção 
de bens passou a ser mecânico. Assim como a 
expansão marítima, as mudanças nos meios de 
produção também redefiniram as forças que 
sustentavam a organização social, bem como o 
equilíbrio entre elas.


As pesquisas sobre substâncias que reagiam à 
ação da luz continuaram. 
No início do século XVIII, foram 
descobertas outras substâncias que também 
eram sensíveis à luz. Os sais de ferro, 
por exemplo, mudam de cor sob a ação da luz, 
criando um pigmento azul-escuro, 
nomeado de \textit{azul da Prússia}. 
As imagens criadas usando essas substâncias, 
embora impressionassem pela definição, 
parecendo-se com imagens refletidas em um 
espelho, eram fugidias; rapidamente toda 
a superfície sensibilizada com a substância 
química tornava-se completamente escura. 
O desafio era encontrar uma maneira de 
interromper a ação da luz, fixando essas imagens.

No início do século XIX, John Herschel 
(Inglaterra, 1792), matemático, químico, 
astrônomo, descobriu a substância 
hipossulfito de sódio e que esta tinha 
a propriedade de dissolver o nitrato de prata. 
Ao publicar seu resultado no Edinburg 
Philosophical Journal ([14] 1819), ele 
descreve o fenômeno como ``a prata derrete 
como açúcar na água''. Na época, pouca atenção 
foi dada às propriedades da nova substância, 
inclusive pelos que investigavam a 
sensibilidade da prata à luz.



\begin{figure}[htb!]
\centering
\includegraphics[width=8cm]{Niépce_Heliograph_1827_Le_Gras.jpg}
\caption{Niépce, Heliografia, 1827}   
\end{figure}


Joseph Nicéphore Niépce (França, 1765) e Louis 
Daguerre (França, 1787) são  nomes que 
entraram na história como  inventores da 
fotografia. Niépce, na década de 1810, 
enfrentou as dificuldades da fixação das 
imagens criadas utilizando sais de prata e 
uma câmara escura, mas não obteve sucesso. 
Em seus experimentos, além da prata, ele 
passou a utilizar o \textit{betume da judéia}, 
um derivado do asfalto natural, que endurece 
sob a ação da luz solar. Em 1822, Niépce, 
em correspondência com seu irmão, anuncia 
que conseguiu produzir a primeira imagem 
utilizando essa substância, tendo como 
suporte uma placa de vidro exposta à luz 
dentro de uma câmara escura. Essa primeira 
imagem não resistiu ao tempo, possivelmente 
pela fragilidade do vidro. A fotografia mais 
antiga, feita por Niépce, utilizando betume 
sobre uma placa de estanho exposta à luz 
dentro de uma câmara escura, hoje encontra-se 
exposta no museu \textit{Harry Ransom Center}, 
Texas -- EUA. Datada de 1827, a placa serviu 
como matriz para a produção de cópias da imagem. 
A técnica criada por Niépce hoje é chamada de 
\textit{heliografia}.


Nos anos seguintes, Niépce também procurou 
aprimorar o uso das lentes acopladas à 
câmara escura, para aperfeiçoar a nitidez das 
imagens e diminuir o tempo de exposição que 
era muito longo, inviabilizando a 
comercialização do invento. Nessa busca, 
Niépce teve contato com Daguerre, que 
também trabalhava na produção de imagens 
utilizando a luz. Os dois usavam o serviço da 
ótica parisiense de Jacques Chevalier (França, 1777), 
que colocou os dois em contato.
Niépce e Daguerre trabalharam juntos com 
experimentações com iodeto de prata e mercúrio. 
Niépce faleceu em 1833. Daguerre continuou as 
pesquisas e, em 1839, anunciou o processo 
fotográfico chamado de \textit{daguerreótipo}. 
Cada imagem feita pelo daguerreótipo era única, 
como uma jóia. A patente do novo aparelho foi 
vendida ao governo francês no mesmo ano, 
que a tornou domínio público.
  Rapidamente, a fotografia ganhou o mundo. [9]


 \begin{figure}[htb!]
\centering
\includegraphics[width=8cm]{Daguerreotype_process (1).jpg}
\caption{Daguerreótipo, Daguerre, 1839}   
\end{figure}
 


A \textit{reprodutibilidade} que caracteriza a 
fotografia foi conseguida a partir do trabalho 
de William Fox Talbot (Inglaterra, 1800), 
que inventou as técnicas do \textit{papel salgado} 
e do \textit{calótipo}, na década de 1830, 
processos fotográficos realizados em duas 
etapas: negativo e positivo [9]. A química 
usada era baseada no nitrato de prata, mas 
a grande novidade era a produção de uma 
primeira imagem em negativo, utilizando 
uma câmara escura, que depois seria usada 
como matriz para a reprodução da mesma 
imagem, agora exposta diretamente à luz do sol, 
processo chamado \textit{fotograma}, 
fotografia sem câmera. Em 1843, Talbot 
também desenvolveu uma técnica de ampliação, 
reproduzindo cópias de tamanho maior do que 
o negativo. Esses métodos de reprodução/ampliação 
foram a base do processo fotográfico até a 
invenção da fotografia digital, mais de 
cem anos depois.

Todos esses experimentos que obtiveram 
sucesso na impressão de imagens usando o 
nitrato  ou o iodeto de prata exposto à 
luz solar, utilizavam o hipossulfito de sódio, 
descoberto por Herschel vinte anos antes, 
para fixar as imagens.



Entre 1833 a 1839, o francês Hercule 
Florence (1804), vivendo em Campinas -- Brasil, 
desenvolveu uma pesquisa inédita e independente, 
buscando imprimir imagens usando a luz. O 
jovem Florence chega ao Brasil em 1824, 
desembarcando no Rio de Janeiro em busca de 
experiências no novo mundo. Entre 1825 e 1829, 
participou da expedição científica Langsdorff, 
como desenhista. A expedição, patrocinada pelo 
governo russo e chefiada pelo barão Georg 
von Langsdorff, médico alemão naturalizado 
russo, tinha o objetivo de aproximar as 
relações entre Rússia e Brasil, recém 
independente de Portugal. A viagem passou 
por Minas, São Paulo e seguiu até a Amazônia. 
Florence também foi o responsável pelo relato 
da expedição ao final da jornada.




No início da década de  1830, já 
estabelecido em Campinas, Florence, 
autodidata e profundamente influenciado 
pelo período de viagem na expedição, pesquisou 
a \textit{zoophonia}, estudo das vozes 
dos animais,  e, para divulgar suas descobertas, 
desenvolveu um método de impressão que ele 
chamou de \textit{poligrafia}. Embora 
independente de Portugal, a estrutura 
econômica e social do Brasil era comparada 
à estrutura feudal, rural e escravocrata, 
do século XV.  A circulação de informações 
era muito limitada. A prensa móvel, 
inventada por Johannes Gutenberg 
(Alemanha, c.1400), em 1450, que tinha 
sido fundamental na divulgação do conhecimento 
produzido durante a revolução científica, 
ainda era artigo raro no Brasil Império. 
Nos anos seguintes, Florence aperfeiçoou 
seu invento, obtendo sucesso na comercialização 
de materiais impressos.

Tendo tido conhecimento das propriedades 
do nitrato de prata sob ação da luz, 
Florence passou a pesquisar a impressão 
usando a substância.
Aproximadamente em 1833, Florence produziu 
rótulos para produtos farmacêuticos já 
utilizando o método de impressão com o 
nitrato de prata inventado por ele. 
Continuou as pesquisas e tentou divulgar 
seus resultados, sem ter conseguido 
reconhecimento, até 1839, quando a notícia 
sobre a invenção de Daguerre foi publicada 
pela imprensa brasileira.


 \begin{figure}[htb!]
\centering
\includegraphics[width=7cm]{Hercule Florence-Acervo-Instituto Moreira Salles.jpg}
\caption{Rótulos farmacêuticos, Florence, c. 1833}   
\end{figure}

Os diários escritos por Florence, descrevendo
as experiências feitas e os resultados 
obtidos no período, inclusive usando a palavra 
\textit{fotografia}, que significa escrita com 
a luz, alguns anos antes de Herschel, que 
oficialmente foi o primeiro a usar palavra 
em 1839, são a documentação histórica dos 
feitos alcançados por ele. Mais de cem anos 
depois, o pesquisador e fotógrafo brasileiro 
Boris Kossoy publicou sua pesquisa sobre vida 
e obra de Hercule Florence em sua tese de 
doutorado (USP, 1979 [18]). Antes disso, 
os feitos de Florence estavam apagados da 
história oficial da fotografia. Esse evento  
permite uma reflexão sobre como a História 
é construída e como o acesso a recursos e 
o poder da narrativa influenciam no registro 
dos acontecimentos.



Em 1850, os fundamentos científicos para a 
produção de uma fotografia estavam bem 
estabelecidos. Sob a influência da expansão 
das indústrias, o êxodo do campo para as 
cidades desencadeou o início de problemas 
estruturais que precarizavam as condições 
de vida dos  habitantes dos centros urbanos. 
Pesquisas científicas para enfrentar os 
novos problemas surgiam. Como exemplo, 
na década de 1860, o microbiologista 
francês Louis Pasteur (1822) desenvolve 
o processo de \textit{pasteurização}, 
que elimina microorganismos causadores de 
doenças, aumentando a segurança alimentar. 
O trem, inventado em 1804, foi fundamental 
na expansão da industrialização, encurtando 
distâncias e tempo, permitindo uma maior 
circulação de mercadorias e pessoas. 
A economia tornou-se centrada na indústria. 
A \textit{burguesia industrial} surge como a 
classe que detém os meios de produção: 
fábrica, máquinas e matéria prima, usando 
como mão de obra os trabalhadores assalariados.


A fotografia se adaptou bem a esse contexto,  
servindo bem à essa nova classe burguesa 
nascida das revoluções iluminista e industrial, 
fundamentalmente urbana. Era a primeira vez 
que pessoas fora da nobreza e da aristocracia, 
parte da sociedade que circundava a nobreza, 
tinham a possibilidade de perpetuar a sua 
própria imagem.

Patenteado pelo fotógrafo francês André Disdéri, 
em 1854, o formato dos ``cartes de visite'' 
(cartões de visita) possibilitaram a produção 
em massa, sintonizado ao ritmo industrial, 
de pequenos retratos que eram colados em 
cartões de papel rígido e depois distribuídos 
entre familiares e amigos. Os cartões de 
visita tornaram-se um importante símbolo de 
status social. [32]



Desde os primeiros anos após a invenção da 
fotografia, esta passou a ser usada como 
ferramenta de investigação e registro da realidade.

O livro ``Photographs of British Algae: Cyanotype 
Impressions'' (Fotografias de algas britânicas: 
impressões em cianotipia, 1843 [1]), de Anna Atkins
(Inglaterra, 1799), é considerado a primeira 
publicação ilustrada com fotografias. A 
produção dos exemplares foi totalmente manual 
e a autora utilizou o método de impressão de 
fotogramas utilizando sais de ferro chamado 
\textit{cianotipia}, inventado por John 
Herschel, em 1842. As imagens azuis das 
algas presentes no livro foram um marco 
nos estudos da botânica pois, até então, 
todo o registro de plantas era feito em 
desenhos, cuja precisão e detalhamento 
dependiam da habilidade dos desenhistas. 
Os livros que resistiram ao tempo, hoje 
se encontram em museus e centros de pesquisa 
e preservação.

O início do fotojornalismo foi marcado 
pela publicação da primeira imagem 
fotográfica em jornal, em 1848. A 
impressão sobre gravura feita a partir de 
um daguerreótipo, foi o registro visual de 
uma rua  de Paris com barricadas, durante 
as revoltas ocorridas na cidade naquele ano. 
Duas imagens feitas por um morador, da 
janela do seu apartamento, foram publicadas 
no  \emph{L'Illustration},  jornal de circulação 
semanal [38].


As observações de corpos celestes, tão 
importantes para o desenvolvimento das 
ciências, também se beneficiaram da invenção. 
A primeira fotografia bem sucedida da lua foi 
feita por John W. Draper (Inglaterra/EUA), 
em Nova Iorque 1840 [31], e a primeira 
fotografia do sol foi feita pelos físicos 
franceses Hippolyte Fizeau e Léon Foucault, 
em 1845. Ambas usando o daguerreótipo. [33]


No início da segunda metade do século XIX, 
a figura do fotógrafo e as funções da 
fotografia faziam parte do funcionamento 
da sociedade industrial. Mas, nas décadas 
seguintes, uma nova etapa da revolução 
industrial iria se impor sobre a ordem 
estabelecida, mudando novamente a organização 
social. A fotografia iria acompanhar as mudanças.



\section{A segunda revolução industrial e a sociedade moderna}



Duas importantes evoluções técnicas permitiram 
novas possibilidades no registro de imagens 
usando a luz, ainda no século XIX.

A primeira delas foi a invenção da fotografia 
colorida.

% Seguindo as pesquisas de Newton sobre óptica, 
% onde ele estabelece a decomposição da luz branca 
% em sete cores, o físico e matemático James C. 
% Maxwell (Escócia, 1831) publicou o artigo 
% ``On the theory of compound colors, and the 
% relations of the colours of the spectrum''
% (Sobre a teoria das cores compostas e as 
% relações das cores do espectro, 1860 [20]), 
% onde demonstra que a luz de qualquer cor 
% pode ser conseguida misturando luzes nas 
% três cores primárias: vermelho (R), verde 
% (G) e azul (B). A teoria de Maxwell se 
% baseia na anatomia e funcionamento do 
% olho humano, que enxerga as cores primárias 
% através de três tipos de células, chamadas 
% cones, localizadas na retina. Em 1854, a 
% pesquisa de Carl Bergmann, anatomista 
% alemão (1814),  mostrava que as células cones, 
% que detectam cores, e as células bastões, 
% que detectam a intensidade luminosa, 
% localizadas na retina eram as responsáveis 
% pela conversão da luz em sinais neurais [3]. 

% A primeira fotografia colorida foi 
% construída por Thomas Sutton (Inglaterra, 1819), 
% em 1861, usando projeção de luz através 
% de filtros nas três cores RGB, a partir 
% de um artigo de Maxwell, publicado em 1855 [44]. 

    
% O processo para a impressão colorida teve 
% sucesso alguns anos depois, quando Louis 
% Ducos du Hanson (França, 1837), patenteou 
% um método de impressão em 1868, baseado 
% nas cores secundárias ciano (C), magenta (M) e 
% amarelo (Y) [9]. Nas décadas seguintes, o 
% sistema foi aperfeiçoado com o uso da 
% sensibilização das emulsões fotossensíveis 
% por corantes. Os sistemas aditivo RGB e 
% subtrativo CMYK, onde K refere-se a cor preta, 
% são utilizados até hoje na construção de 
% imagens nas telas dos aparelhos digitais e 
% nas impressões gráficas, respectivamente.

Seguindo as pesquisas de Newton sobre óptica, onde ele estabelece a decomposição da luz
branca em sete cores, o físico e matemático James C. Maxwell (Escócia, 1831) publicou o
artigo ``On the theory of compound colors, and the relations of the colours of the spectrum''
(Sobre a teoria das cores compostas e as relações das cores do espectro, 1860 [20]), onde
demonstra que a luz de qualquer cor pode ser conseguida adicionando luzes nas três cores:
vermelho (R), verde (G) e azul (B). A teoria de Maxwell se baseia na anatomia e funcionamento
do olho humano, que enxerga cores através de três tipos de células, chamadas cones,
localizadas na retina. Em 1854, a pesquisa de Carl Bergmann, anatomista alemão (1814),
mostrava que as células cones, que detectam cores, e as células bastões, que detectam a
intensidade luminosa, localizadas na retina, eram as responsáveis pela conversão da luz em
sinais neurais [3].

A primeira fotografia colorida foi construída por Thomas Sutton (Inglaterra, 1819), em 1861,
usando projeção de luz através de filtros nas três cores RGB, a partir de um artigo de Maxwell,
publicado em 1855 [44].

O processo para a impressão colorida teve sucesso a partir de 1868, quando Louis Ducos du
Hanson (França, 1837) patenteou um método utilizando tintas nas cores: ciano (C), magenta
(M) e amarelo (Y), sobre papel branco [9], baseado na propriedade dos materiais, nesse caso
específico do papel e das tintas, absorverem partes do espectro da luz e refletirem outras. A
mesma propriedade era a base das técnicas usadas pelos artistas pintores na aplicação de
tintas e pigmentos naturais nas cores vermelho, azul e amarelo sobre madeira, pedra e,
posteriormente, sobre telas. A utilização das cores CMY mostrou-se mais eficiente na
construção de uma maior gama de cores, em relação aos resultados conseguidos utilizando as
cores RBY, aumentando assim a qualidade da imagem impressa.

Atualmente, os sistemas \textit{aditivo}, utilizando luzes nas cores RGB, e \textit{subtrativo},
utilizando tintas nas cores CMYK, onde K refere-se a cor preta, são empregados na
construção de imagens nas telas dos aparelhos digitais e nas impressões gráficas,
respectivamente.

Antes da popularização da impressão colorida, 
que ocorreu já no século XX, a coloração 
das imagens fotográficas era feita manualmente. 
Muitos artistas pintores passaram a trabalhar 
com fotografia e alguns voltaram seu interesse 
para esse tipo de atividade, que se transformou 
em grande sucesso comercial. 
A relação da fotografia, recém inventada, e a 
arte, que na época era a chamada arte acadêmica: 
desenho, gravura, pintura, escultura, arquitetura; 
ainda estava iniciando. 

No \textit{impressionismo}, movimento artístico 
nascido na França, na década de 1870, os pintores 
abdicaram do dever de registrar a realidade. 
Agora, a fotografia podia fazer essa tarefa. 
Também o físico alemão Hermann von Helmholtz (1821), 
interessado em estabelecer teorias matemáticas 
sobre a percepção visual, afirmou em seu livro 
``Handbuch der Physiologischen Optik'' 
(Tratado sobre óptica fisiológica, 1867 [13]), 
que uma perfeita representação da natureza 
seria impossível, pois a escala de pigmentos 
é infinitamente menor do que a escala da luz, 
estabelecendo limites técnicos para a 
precisão exigida das obras de arte. A 
partir daquele momento, os artistas estavam 
livres para representar suas  sensações 
sobre natureza, sociedade e outros temas, 
através da sua arte.
    
A impressão fotográfica usando a substância 
platina contribuiu para que a própria 
fotografia passasse a ser considerada uma 
forma de arte.
    
O contato dos europeus com a platina 
aconteceu na América do Sul, no século XVI, 
quando encontraram artefatos religiosos e 
utilitários produzidos por povos originários 
da região onde hoje se localiza a divisa 
entre Equador e Colômbia. A fabricação das 
peças era feita utilizando a técnica de 
\textit{sinterização}, transformando o metal 
em pó em uma massa sólida sem fundi-lo 
completamente. A técnica só seria totalmente 
dominada pelos europeus durante a revolução 
industrial.   
    
A primeira patente sobre a impressão 
fotográfica usando platina foi registrada 
em 1873, por William Willis (Inglaterra, 1841). 
No final da década de 1880, o processo estava 
consolidado. O valor artístico desse tipo 
de impressão se relaciona com o fato de que a 
emulsão de platina é aplicada diretamente no 
papel, diferente da prata que precisa ser 
acrescentada a um meio gelatinoso que 
permanece sobre o papel, sem penetrá-lo. 
Assim, a imagem impressa com a platina 
adquire profundidade, alcançando uma ampla 
gama de tonalidades de cinza. A impressão 
em platina é a mais durável dentre todos os 
tipos de impressão pela estabilidade física 
e química do metal. A estabilidade da platina, 
além da alta condutibilidade elétrica, tornou-a 
importante para a indústria militar no século XX. 
Com o aumento da demanda, o preço do metal 
tornou-se excessivamente caro, restringindo o 
seu uso para a impressão fotográfica. [12]

O segundo grande avanço técnico da fotografia, 
na segunda metade do século XIX, foi a invenção 
da primeira máquina fotográfica portátil, 
patenteada em 1888, por George Eastman (EUA, 1854) 
e comercializada por sua empresa Kodak, fundada 
em 1884, em sociedade com Henry Strong (EUA, 1838). 

No final do século XIX, a invenção de Eastman 
tinha tornado a fotografia um produto industrial e 
comercial de grande sucesso. 

A promessa na publicidade da nova mercadoria era
\begin{center}
``Você aperta o botão e nós fazemos o resto.''
\end{center}


\begin{figure}[htb!]
\centering
\includegraphics[width=8cm]{You_press_the_button,_we_do_the_rest_(Kodak).jpg}
\caption{Propaganda Kodak, 1888}   
\end{figure}

A frase é quase uma antítese perfeita do 
equilíbrio proposto por Galileo em seu 
método científico.


A máquina fotográfica portátil era uma pequena 
câmara escura, acoplada com uma lente fixa e 
munida com um filme fotográfico de cem exposições. 
Não tinha visor. Após fazer o registro das 
imagens, a máquina era levada ao laboratório da 
empresa, para que o filme fosse tratado 
quimicamente e, em seguida, as imagens 
fossem impressas, empregando o mesmo sistema 
negativo-positivo inventado cinquenta anos antes.
 
O filme fotográfico utilizado na câmara também 
era produzido pela empresa. No ano de fundação, 
a Kodak lançou o primeiro filme fotográfico 
usando como suporte o \textit{celulóide}, o 
primeiro material plástico fabricado com 
sucesso comercial,
patenteado pelo engenheiro John Hyatt 
(EUA, 1837), em 1872. [36]
 
Além da facilidade do manuseio do aparelho, 
a produção em massa da primeira câmara 
fotográfica portátil permitiu que o custo 
da produção fotográfica diminuísse, ajudando 
a tornar a fotografia popular. E, junto 
com a popularização, criou-se uma distância 
entre o processo técnico da produção da 
imagem, baseado em teorias científicas, e 
o indivíduo que ``cria'' a imagem apertando um 
botão. Antes, a figura do fotógrafo, como o 
detentor do \textit{saber fazer} estava 
presente, orquestrando a condução do processo. 
Nesse momento, como é característico da 
produção industrial, o produto final 
\textit{fotografia} se afasta da linha 
de produção, dos trabalhadores e dos 
saberes envolvidos no processo, fenômeno 
que só se acentuou, desde então.

As pesquisas sobre a luz permitiram a descoberta 
de fenômenos ``semelhantes à luz'', mas fora do 
espectro visível aos olhos humanos. Em 1800, o 
astrônomo William Herschel (Alemanha, 1738), 
descobriu a onda de luz infravermelha. 
Herschel foi o descobridor do planeta Urano e 
também foi o primeiro a investigar o movimento 
do sol no espaço. Pouco tempo depois, em 1801, 
motivado pela descoberta dos raios infravermelhos, 
Johann Ritter (Alemanha, 1776), pesquisando 
sobre sais de prata, descobriu os raios 
ultravioletas. 
A descoberta do raio X aconteceu em 1895, 
pelo físico alemão Wilhelm Roentgen (1845) 
que, fazendo experimentos sobre a luminescência 
de materiais utilizando uma câmara escura, 
observou que os raios gerados por um tubo 
de Crookes, aparelho que gera raios de elétrons, 
tinham a propriedade de atravessar alguns 
materiais, mas não outros. A primeira 
imagem produzida usando raios x, chamada 
\textit{radiografia}, que é precisamente 
uma fotografia utilizando esse tipo de raio, 
foi a mão da esposa do cientista posicionada 
entre o tubo e a câmara escura com um papel 
fotográfico dentro. A radiografia permitiu, 
pela primeira vez, a visualização do interior 
do corpo humano, sem a necessidade de cirurgia. [26]

O estudo do movimento dos corpos foi outro tema 
de investigação que se beneficiou do uso da 
fotografia.

\begin{figure}[htb!]
\centering
\includegraphics[width=8cm]{The_Horse_in_Motion_high_res.jpg}
\caption{O cavalo em movimento, Muybridge, 1878}   
\end{figure}

O aparelho \textit{zoopraxiscópio}, tendo como 
tradução livre do grego ``aparelho para observar 
a vida na prática'', inventado em 1879, pelo 
fotógrafo inglês Eadweard J. Muybridge (1830) [40], 
projetava uma série de fotografias de um objeto 
em movimento, feitas em vários instantes e em 
um curto intervalo de tempo. A projeção rápida 
da sequência de fotografias dava ao observador a 
ilusão do movimento do objeto.
A técnica usada para o registro da série de 
imagens chama-se \textit{cronofotografia} e 
começou a ser desenvolvida logo após à invenção. 

Joseph Plateau, matemático e físico belga (1801), 
em sua tese de doutorado [25], foi o primeiro 
a estudar a duração da imagem na retina. Em 
1832, décadas antes de Muybridge, Plateau 
inventou o aparelho \textit{fenacistoscópio}, 
um instrumento que criava a ilusão de movimento, 
utilizando desenhos de um mesmo objeto em 
diferentes posições. Esse aparelho é considerado 
o primeiro dispositivo de animação. 

O desenvolvimento desses experimentos e 
aparelhos, mais tarde, culminaria na 
invenção do cinema.

Na segunda metade do século XIX, a 
\textit{indústria da energia} e  a 
\textit{indústria das comunicações} se 
expandiram, tornando-se grandes conglomerados. 
Concentrando riqueza e poder,  moldaram o 
desenvolvimento científico e tecnológico, 
influenciando  nos rumos da sociedade.

A principal fonte de energia durante a 
primeira fase da revolução industrial foi 
o \textit{carvão mineral}, que alimentava 
as máquinas à vapor e os fornos das indústrias 
metalúrgicas.

Com a primeira patente do motor à combustão 
interna, registrado em Londres, em 1854, 
pelos engenheiros italianos Eugenio Barsanti 
(1821) e Felice Matteucci (1804), e os 
posteriores aprimoramentos da invenção, o 
\textit{petróleo}, fonte de energia química, 
passou a ser o principal combustível dos 
motores. Desde então, a extração de petróleo 
tornou-se fator decisivo no desenvolvimento 
industrial dos países, e na consequente 
geração de riquezas. Entre os países que 
mais produziram petróleo no século XIX 
estavam os Estados Unidos, a Rússia e 
o Azerbaijão. A indústria petroquímica 
passou a ser um importante ramo da indústria 
química, sendo responsável pela produção 
dos derivados do petróleo, como combustíveis 
e plásticos.

Podemos dizer que a história do uso da 
eletricidade como fonte de energia para o 
funcionamento de motores, se inicia na 
antiguidade, quando tempestades e peixes 
que ``dão choque'' despertavam a curiosidade 
de todos e eram investigados pelos que se 
dedicavam à busca de explicações. 
Também não seria exagero dizer que, 
assim como a luz, a eletricidade e o 
magnetismo foram objetos de investigação 
dos principais cientistas até aquele momento. 
No século XIX, as pesquisas sobre a eletricidade, 
assim como as suas interações com a luz e o 
magnetismo ganharam impulso.

A relação entre eletricidade e magnetismo foi 
descoberta em 1819, quando Hans Christian Oersted 
(Dinamarca, 1777), identificou que a posição 
da agulha magnética de uma bússola sofria 
interferência ao se aproximar de um fio 
por onde passava energia elétrica. [24]

Em 1831, Michael Faraday (Inglaterra, 1791) 
descobriu o fenômeno  da \textit{indução 
eletromagnética}, a geração de energia 
elétrica através da variação de um campo 
magnético. [10]


Edward Becquerel (França, 1820), pesquisando 
as propriedades do brometo de prata, em 1839, 
descobriu o chamado \textit{efeito fotovoltaico}, 
a criação de corrente elétrica em um material, 
após a sua exposição à luz. [2]

Em 1873, os resultados das pesquisas realizadas 
pelo físico e matemático James Maxwell, entre 
1850 e 1870, foram condensadas no livro 
``A treatise on Electricity and Magnetism'' 
(Tratado sobre eletricidade e magnetismo [21]).  
Nessa obra, estão estabelecidas as equações 
que unificam os três campos: eletricidade, 
magnetismo e luz. Segundo a teoria de Maxwell, 
a luz pode ser considerada como uma propagação 
de \textit{ondas eletromagnéticas}. 
Maxwell é considerado o terceiro mais 
influente físico da história, Newton e 
Einstein ocupando as duas primeiras posições.


A existência das ondas eletromagnéticas, previstas 
teoricamente pelas equações de Maxwell, 
foram detectadas, experimentalmente, em 1887, 
por Heinrich Hertz (Alemanha, 1857). Na sua 
pesquisa, Hertz descobriu as ondas de rádio, 
que, mais tarde, seriam utilizadas para 
transmitir som.

Nos anos seguintes, as ondas eletromagnéticas 
se tornaram a principal ferramenta das 
duas indústrias: energia e comunicações.

O uso da energia elétrica pelas indústrias 
foi possível a partir da invenção do \textit{dínamo}, 
aparelho que converte energia mecânica em energia 
elétrica, através da indução eletromagnética. 
Um modelo de dínamo que poderia ser usado 
efetivamente pela indústria foi desenvolvido 
por Henry Wilde (Inglaterra, 1833). Os 
resultados de sua pesquisa foram apresentados 
à Royal Institution, por Michael Faraday, 
em 1866 [28]. O primeiro trem com motor elétrico 
foi lançado pela empresa alemã  Siemens \& Halske, 
em 1879.
 
Com a invenção do motor elétrico, começou 
a adaptação das indústrias ao novo tipo de 
energia, 
%Assim, iniciava-se a \textit{segunda 
inicianndo-se a \textit{segunda 
fase da revolução industrial}.

Além do funcionamento das máquinas, a energia 
elétrica passou a ser usada para a geração de 
luz. Thomas Alva Edison (EUA, 1847), inventor 
e homem de negócios, era um dentre os muitos 
inventores, de vários lugares do mundo, que 
buscavam a criação de um dispositivo que 
transformasse energia elétrica em luz.  
Edison patenteou a lâmpada elétrica 
incandescente em 1880.

O próximo objetivo de Thomas Edison foi 
construir um sistema de produção e 
distribuição de energia elétrica em 
grande escala para o uso em iluminação. 
Em 1882, o sistema foi inaugurado, atendendo 
consumidores de Nova Iorque, sendo a 
fonte de energia elétrica uma usina 
termoelétrica, alimentada por carvão mineral. 
A primeira hidrelétrica  foi inaugurada no 
mesmo ano, no rio Fox, nos Estados Unidos. 
Edison era um dos envolvidos no projeto.

A corrente elétrica distribuída pela empresa 
de Edison era contínua, o que impedia a 
transmissão para longas distâncias. Nikolas 
Tesla (Sérvia, 1856), inventor do 
\textit{motor por indução}, também foi o 
inventor da \textit{corrente alternada}, 
que tinha a vantagem de permitir a transmissão  
por longas distâncias. Em sociedade com Tesla, 
George Westinghouse, inventor e empreendedor 
industrial americano (1846),  passou a 
disputar o fornecimento de energia com Edison, 
episódio que é conhecido como ``a guerra das 
correntes'' [35]. No final do século, a 
corrente alternada tinha sido adotada como 
o padrão de transmissão de energia elétrica. 

A empresa Westinghouse Electric Manufacturing 
Company, fundada por George Westinghouse em 1886, 
agora, no século XXI, atua na geração de energia 
nuclear. Edison participou da fundação das 
empresas Edison Lamp Company (lâmpadas), 
Edison Machine Works (dínamos e motores elétricos), 
Bergman \& Company (luminárias, tomadas e 
outros acessórios para iluminação elétrica), 
Edison Illuminating Company (estações de 
geração de energia elétrica). As empresas de 
Edison foram precursoras da empresa General 
Electric. Recentemente, a \textit{GE} foi 
reestruturada e dividida em três empresas 
que produzem: motores para aviões, 
turbinas eólicas e tecnologias para a 
produção de energia renovável, aparelhos 
tecnológicos voltados para a área de saúde. [34]

A indústria da comunicação foi completamente 
transformada com a invenção do \textit{telégrafo}. 
A invenção atualizou a troca de informações, 
conhecimentos e sentimentos para a era industrial.

A tecnologia empregada na invenção envolvia 
eletricidade e magnetismo. O primeiro 
telégrafo eletromagnético foi construído 
pelos alemães Carl Friedrich Gauss 
(sim, o matemático!) e o físico Wilhelm Weber, 
em 1833. Os dois, que pesquisavam juntos 
sobre magnetismo, conectaram o observatório e 
o instituto de física da Universidade de 
Göttingen, onde trabalhavam, com fios de 
transmissão.

Em 1844, Samuel Morse (EUA, 1791) inaugurou 
uma linha de telégrafo entre as cidades de 
Washington DC e Baltimore, nos Estados Unidos, 
construída ao longo da estrada de ferro  
que ligava as cidades. Além da versão 
comercial do telégrafo, Morse criou o 
código que se tornou padrão para as 
transmissões das mensagens de texto. Em 1845, 
fundou a Magnetic Telegraph Company com o 
objetivo de construir linhas de telégrafo 
ao longo do país. As tentativas de expandir 
os negócios instalando cabos submarinos entre 
a Europa e a América do Norte foram iniciadas 
na década de 1850. O feito foi alcançado em 
1866, tornando possível a comunicação entre 
os dois continentes através do telégrafo. 


A próxima etapa na evolução do envio de 
dados envolveu a transmissão do som. 
Alexander Graham Bell (EUA, 1847) patenteou 
o telefone eletromagnético em 1876. A Bell 
Telephone Company foi fundada em 1877 e, 
em 1899, transformou-se em AT\&T --- 
American Telephone and Telegraph Company. 
Em 1915, foi feita a primeira ligação 
telefônica transcontinental entre Boston, 
na costa atlântica, e São Francisco, 
costa pacífica, dos Estados Unidos. 
O primeiro cabo telefônico submarino  
foi instalado em 1956, ligando Inglaterra e 
Canadá.


Além de Bell, os inventores  Edison e Tesla 
também  desenvolveram aparelhos envolvendo 
a transmissão e gravação do som. Tesla 
pesquisou a transmissão de som sem o uso 
de fios, utilizando as ondas eletromagnéticas 
de rádio. A invenção do \textit{rádio}, por 
Guglielmo Marconi (Itália, 1874), aconteceu 
em 1896. Edison, em 1877, inventou o 
\textit{fonógrafo} que gravava e reproduzia som. 
Dez anos depois, Emile Berliner (Alemanha, 1851) 
patenteou o \textit{gramofone}, que reproduzia 
sons gravados em discos planos. Ao contrário do 
fonógrafo, a estrutura de funcionamento do 
gramofone permitia sua produção pela indústria.


A transmissão de imagens foi o passo seguinte 
no desenvolvimento da transmissão de dados. 
Na primeira década do século XX,  Arthur Korn, 
inventor, matemático e físico (Alemanha, 1870), 
desenvolveu o primeiro dispositivo de 
varredura fotoelétrica dando origem  à 
\textit{telefotografia} [45]. Vamos  continuar 
essa história na segunda parte do texto.


Diante desse fluxo intenso de invenções 
ocorridas nesse período, é importante questionarmos
de onde vinha o dinheiro que mantinha o ritmo 
das pesquisas. Os apoios financeiros para a 
invenção do telégrafo, do telefone e do sistema 
de produção e distribuição de energia elétrica, 
são bem ilustrativos da diversidade de situações 
que poderiam acontecer em qualquer país 
industrializado ao final do século XIX. 

% A pesquisa de Morse contou com o financiamento 
% do congresso americano, por intermédio do 
% parlamentar e advogado Francis O. S. Smith 
% (USA, 1806) [43], que posteriormente tornou-se 
% sócio de Morse. Era o início do investimento 
% público em pesquisa aplicada à indústria [REF].  
A pesquisa realizada por Morse contou com o financiamento
do congresso americano, conseguido por intermédio do parlamentar
e advogado Francis O. S. Smith (USA, 1806) que, 
posteriormente, tornou-se sócio de Morse.
Ao final do século XIX, investigar as possibilidades 
de aplicações de descobertas científicas
para o desenvolvimento e aprimoramento das indústrias 
tornou-se uma questão estratégica
para os países, passando assim a receber investimento público [43].
Edison, cuja família paterna era originária 
da Holanda, pode ser considerado um representante 
do ``sonho americano'',  por ter conseguido 
sucesso pelo próprio trabalho. Mas, é importante 
destacar que a Electric Light Company, o braço 
financeiro das empresas criadas a partir 
das invenções de Edison, era associada à 
J. P. Morgan (EUA, 1837), banqueiro de 
investimentos, herdeiro de uma das primeiras 
famílias inglesas a chegar na colônia americana [39]. 
Bell, nascido em uma família de professores da 
Escócia, iniciou sua trajetória nos Estados Unidos 
também como professor, conseguindo sucesso 
financeiro suficiente para manter suas pesquisas 
iniciais. Anos mais tarde, o pai de uma de suas 
alunas, Gardiner Greene Hubbard, tornou-se amigo 
e patrocinador dos experimentos de Bell. 
Hubbard, que viria a ser sogro de Graham Bell, 
era neto de colonizador inglês estabelecido 
em Nova Iorque e com fazendas de algodão, 
café e açúcar na América do Sul [37].

Analisando o contexto que envolveu o avanço 
tecnológico industrial no final dos anos 1800, 
podemos perceber que a força desse desenvolvimento 
foi fruto do espírito inovador e do trabalho 
de muitos  pesquisadores-inventores, mas 
também foi alimentado pelas riquezas acumuladas 
nos séculos anteriores, em economias baseadas 
no feudalismo e no colonialismo.

No final do século XIX, a sociedade e 
a economia industrializadas exigiam novos 
desenvolvimentos científicos, com o objetivo 
de expandir a indústria e torná-la mais eficiente. 
Os nomes que se destacaram na evolução das 
indústrias de energia e das comunicações, 
também tiveram relevância na gerência da 
continuidade das pesquisas voltadas à 
indústria, institucionalizando a área de 
pesquisa científica que passou a ser chamada 
\textit{ciências aplicadas}.


Nesse sentido, Thomas Edison criou o 
\textit{Menlo Park Street Laboratory}, considerado 
o primeiro laboratório industrial.  As invenções 
desenvolvidas no Menlo Laboratory eram de 
propriedade de Edison, que possui mais de 
mil patentes registradas em seu nome. 
Em 1880, Edison foi um dos financiadores do 
lançamento da revista científica 
\textit{Science} [42]. Alguns anos depois, 
Graham Bell e seu sogro Gardiner Hubbard 
compraram os direitos sobre a revista. 
Hubbard também foi um dos fundadores da 
\textit{National Geographic Society}, e o 
seu primeiro presidente, seguido por seu 
genro Graham Bell [41]. 

Para apoiar as pesquisas que realizava 
com o som, no início da década de 1880, 
Bell criou o \textit{Volta Laboratory and Bureau}. 
Em nome do laboratório estão registradas as 
patentes da invenção do telefone. Em 1915, 
a empresa AT\&T, participante do conglomerado 
Bell System, inaugurou o \textit{Bell 
Telephone Laboratories}, que tinha 
como objetivo consolidar as pesquisas 
na área de comunicações e suas ramificações. 
Funcionava com 3.600 funcionários,  
num prédio de 37.000 $m^2$, em Nova Iorque. 
Nas décadas seguintes, as pesquisas desenvolvidas 
pelo Bell Labs definiriam os rumos das novas 
tecnologias envolvendo comunicação. [30]



Diante da importância dos recursos naturais, como  
água, carvão, petróleo, ferro e outros metais, 
para a existência das indústrias, os países, 
que no final do século XIX tinham alcançado a 
consolidação da sua produção industrial: 
Inglaterra, França, Bélgica, Alemanha, Itália, 
Estados Unidos e Japão; passaram a disputar 
os recursos naturais localizados em outras 
regiões do planeta. Era o início de um novo 
período de colonização. Na ata da 
``Conferência de Berlim'', em 1884, organizada 
pelo chanceler do Império Alemão, Otto von Bismarck, 
está escrito o objetivo da reunião: 
``regulamentar a liberdade do comércio nas 
bacias do Congo e do Níger, assim como novas 
ocupações territoriais sobre a costa ocidental 
da África'' [29]. 


\begin{figure}[htb!]
\centering
\includegraphics[width=8cm]{inauguracao-caboclo.jpeg}
\caption{Inauguração Monumento ao 2 de Julho, 1895, fotografia, autor desconhecido}   
\end{figure}

Na virada do século, a fotografia ganhou uma 
nova função: a de testemunho histórico. 
Das atrocidades cometidas pela colonização na África, 
às precárias condições de trabalho nas fábricas, 
passando pelo registro da natureza e de modos de 
vida,  que não resistiriam às transformações 
impostas pela industrialização, a fotografia foi  
uma ferramenta de documentação e denúncia, 
uma espécie de guardiã da memória visual.


Ao final do século XIX, a estrutura social tinha 
se ``modernizado'', tentando se adaptar às muitas 
e tão profundas transformações. Transformações 
que, de maneira semelhante ao que aconteceu no 
início do séc XVI, determinariam os rumos da 
humanidade do século seguinte, reverberando até 
o século XXI. 


\section{A seguir...}

Na segunda parte do texto, vamos ver como o 
efeito fotovoltaico foi fundamental na 
transmissão de imagens através das ondas 
eletromagnéticas. Depois, vamos acompanhar os 
movimentos históricos e os desenvolvimentos 
científicos que desencadearam a criação do 
mundo digital e como toda a estrutura industrial, 
inclusive as comunicações e a produção de 
imagens usando a luz, se adaptaram a esse novo 
mundo, até chegar à internet, ao telefone celular 
e à inteligência artificial.

Investigaremos como as teorias matemáticas 
são fundamentais para fazer todo esse mundo 
funcionar.  Enquanto estamos distraídos, 
os rumos do futuro estão sendo traçados 
nos \textit{espaços latentes} das inteligências 
artificiais, que, segundo o \emph{Chat GPT4}, 
são ``subconscientes matemáticos'' das IAs.

Como estudantes, professores e pesquisadores 
das ciências e, principalmente, como indivíduos 
do XXI, precisamos refletir sobre como o 
conhecimento científico e as novas tecnologias 
desenvolvidas a partir dele estão sendo usadas 
para moldar as próximas décadas e séculos, o 
nosso futuro.

\vspace{0.2cm}
\nocite{*}
%\vfill

%\bibliography{main}
%\section{Bibliografia}
{\fontsize{11}{11}\selectfont
\begin{thebibliography}{99}
\bibitem{[1]} ATKINS, Anna. \textit{Photographs of British Algae: Cyanotype Impressions}. London: s.n., 1843.
\bibitem{[2]} BECQUEREL. A. E., \textit{Mémoire sur les Effects d’Electriques Produits Sous l’Influence des Rayons Solaires} (Relatório sobre os efeitos das ondas elétricas produzidas sob a influência dos raios solares.), Comptes Rendus de l’Academie des Sciences, Vol. 9, 1839, pp. 561-567.
\bibitem{[3]} BERGMANN, C. 1857. \textit{Anatomisches und Physiologisches über die Netzhaut des Auges} (Informações anatômicas e fisiológicas sobre a retina do olho). Zeitschrift für rationelle Medicin 3:83–108.
\bibitem{[4]} COPÉRNICO, Nicolau. \textit{Revolutions of the Celestial Spheres}. Manuscrito, 1543. 
\bibitem{[5]} DA VINCI, Leonardo. \textit{The Notebooks of Leonardo Da Vinci}. New York: Dover Publications, 1970.
\bibitem{[6]} DESCARTES, René. \textit{Discours de la Méthode pour bien conduire sa raison, et chercher la vérité dans les sciences}. Leiden: Jan Maire, 1637.
\bibitem{[7]} DESCARTES, René. \textit{La Dioptrique}. Leiden: Jan Maire, 1637.
\bibitem{[8]} DIJKSTERHUIS, Fokko Jan. \textit{Stevin, Huygens and the Dutch Republic (The Golden Age of Mathematics)}. Nieuw Archief voor Wiskunde, v. 9, n. 2, 2008. 
\bibitem{[9]} EDER, Josef Maria. \textit{History of Photography}. Dover Publications. New York:  1945.
\bibitem{[10]} FARADAY, Michael. \textit{Experimental Researches In Electricity}.  Vol. 1.. Printed by Taylor and Francys.  Londres, 1839
\bibitem{[11]} GALILEI, Galileo. \textit{Dialogue Concerning the Two Chief World Systems}. Florence: Giovanni Battista Landini, 1632.
\bibitem{[12]} HAFEY, John \& SHILLEA, TOM. \textit{The Platinum Print}, Graphic Art Research Center, Rochester Institute of Technology ,1979 
\bibitem{[13]} HELMHOLTZ, Hermann von. \textit{Treatise on Physiological Optics}, Editado por: James P.C. Southall, Optical Society of America, 1925 
\bibitem{[14]} HERSCHEL, John Frederick William. \textit{On the hyposulphurous acid and its compounds}. Edinburgh Philosophical Journal, Edinburgh: Archibald Constable \& Co., v. 1, 1819, p. 8–29; 396–400.
\bibitem{[15]} HUYGENS, Christiaan. \textit{Traité de la lumière}. Leiden: Pierre van der Aa, 1690.
\bibitem{[16]} JUN, Wenren. \textit{Ancient Chinese Encyclopedia of Technology}. Translation and Annotation of Kaogong Ji, The Artificers' Record. London: Taylor \& Francis, 2014.
\bibitem{[17]} KEPLER, Johannes. \textit{Astronomiae Pars Optica}. Augsburg: David Franck, 1604.
\bibitem{[18]} KOSSOY, Boris. \textit{Hercule Florence: A Descoberta Isolada da Fotografia no Brasil}. São Paulo: Edusp, 2007.
\bibitem{[19]} KOSSOY, Boris. \textit{Dicionário histórico-fotográfico brasileiro: fotógrafos e ofício da fotografia no Brasil (1833–1910)}. São Paulo: Instituto Moreira Salles, 2002.
\bibitem{[20]} MAXWELL, James C., \textit{On the theory of compound colours, and the relations of the colours of the spectrum}.  Phil. Trans. R. Soc. (1860) (150): 57–84.
\bibitem{[21]} MAXWELL, James C. \textit{A Treatise on Electricity and Magnetism}. Oxford Press. 1873
\bibitem{[22]} NEWTON, Isaac. \textit{Philosophiae Naturalis Principia Mathematica}. London: Joseph Streater, for the Royal Society, 1687.
\bibitem{[23]} NEWTON, Isaac.\textit{ Opticks: or, A Treatise of the Reflexions, Refractions, Inflexions and Colours of Light}. London: Samuel Smith \& Benjamin Walford, 1704.
\bibitem{[24]} ØRSTED, H.C. \textit{Experimenta Circa Effectum Conflictus Electrici in Acum Magneticam}, Hafniae, Schultz, 1820. 
\bibitem{[25]} PLATEAU, Joseph. \textit{Dissertation sur quelques propriétés des impressions produites par la lumière sur l'organe de la vue}, présentée et soutenue, sous le rectorat de Mr J. Kinker, à la Faculté des sciences de l'Université de Liège,  mai 1829,  pour obtenir le grade de docteur en sciences mathématiques et physiques. 
\bibitem{[26]} RÖNTGEN. Wilhelm Conrad. \textit{Ueber Eine Neue Art von Strahlen} (Sobre uma nova espécie de Raios), Sitzungsberichte Wurzberger der Physik-medic, 1895
\bibitem{[27]} SABRA, A. I. \textit{The Optics of Ibn al-Haytham. Books I–II–III: On Direct Vision}. London: The Warburg Institute, University of London, 1989. 2 vols. (Studies of the Warburg Institute, v. 40).
\bibitem{[28]} WILDE, Henry.\textit{ Experimental researches into electricity and magnetism}, Proceedings of the Royal Society, 1866, p. 107-111
\bibitem{[29]} ATA GERAL DA CONFERÊNCIA DE BERLIM, Berlim, redigida em 26 de fevereiro de 1885. Disponível em: 
\url{https://mamapress.wordpress.com/wp-content/uploads/2013/12/conf_berlim.pdf}
Acesso em: 01 de dezembro de 2025
\bibitem{[30]} BELL Telephone Laboratories. Página no Wikipedia
\url{https://en.wikipedia.org/wiki/Bell_Labs}
Acesso em: 01 de dezembro de 2025
\bibitem{[31]} John William DRAPER. Daguerreotype of the Moon. Nova York: s.n., 1840–1841.
\url{https://www.metmuseum.org/art/collection/search/789162}
Acesso em: 01 de dezembro de 2025
\bibitem{[32]} André DISDERI. Entrada biográfica em Brasiliana Fotográfica. Rio de Janeiro: Fundação Biblioteca Nacional; Instituto Moreira Salles, Disponível em: 
\url{https://brasilianafotografica.bn.gov.br/?p=3873}
Acesso em: 01 de dezembro de 2025
\bibitem{[33]} ESA; Foucault, Hippolyte; Fizeau, Louis. First Photograph of the Sun. Paris: Observatoire de Paris, 1845 Disponível em: 
\url{https://www.esa.int/ESA_Multimedia/Images/2004/03/First_photo_of_the_Sun_1845}
Acesso em: 01 de dezembro de 2025
\bibitem{[34]} General Electric Company. Página no Wikipedia: 
\url{https://en.wikipedia.org/wiki/General_Electric}	
Acesso em: 01 de dezembro de 2025
\bibitem{[35]} Guerra das correntes. Página no Wikipedia: 
\url{https://en.wikipedia.org/wiki/War_of_the_currents}
Acesso em: 01 de dezembro de 2025
\bibitem{[36]} John HYATT. Patente n US133229A Disponível em: 
\url{https://patents.google.com/patent/US133229A/en} Acesso em: 01 de dezembro de 2025
\bibitem{[37]} Gardiner Greene HUBBARD. Página no Wikipedia: 
\url{https://en.wikipedia.org/wiki/Gardiner_Greene_Hubbard} 
Acesso em: 01 de dezembro de 2025
\bibitem{[38]} L'Illustration: Journal Universel. Paris: Bureau du Journal L'Illustration, v. 11, s.d. Disponível em: 
\url{https://books.google.com.br/books?id=_MlLAAAAcAAJ&printsec=frontcover&source=gbs_ge_summary_r&cad=0#v=onepage&q&f=true}	 
Acesso em: 01 de dezembro de 2025
\bibitem{[39]} John Pierpont MORGAN.  Pàgina no Wikipedia: 
\url{https://en.wikipedia.org/wiki/J._P._Morgan}
Acesso em: 01 de dezembro de 2025
\bibitem{[40]} Eadweard J. MUYBRIDGE. Página no Google Arts \& Culture: 
\url{https://artsandculture.google.com/entity/m0gc57?hl=pt} 
Acesso em: 01 de dezembro de 2025
\bibitem{[41]} National Geographic Society. Página no Wikipedia: 
\url{https://en.wikipedia.org/wiki/National_Geographic_Society#cite_note-9}	
Acesso em: 01 de dezembro de 2025
\bibitem{[42]} SCIENCE, Revista Científica. Site: 
\url{https://www.science.org/content/page/about-science-aaas}
Acesso em: 01 de dezembro de 2025
\bibitem{[43]} Francis Ormand J. SMITH. Página no Wikipedia: 
\url{https://en.wikipedia.org/wiki/Francis_Ormand_Jonathan_Smith} 
Acesso em: 01 de dezembro de 2025
\bibitem{[44]} Thomas SUTTON . The British Journal of Photography, 1875 Vol. XXII (London: Henry Greenwood), pp. 210-212.
\url{https://archive.org/details/britishjournalof58londuoft/page/n1093/mode/2up} 
Acesso em: 01 de dezembro de 2025
\bibitem{[45]} The New York Times. \textit{Sending Photographs by Telegraph}, Sunday Magazine, 20 September 1907, p. 7. Disponível em:
\url{https://en.wikisource.org/wiki/The_New_York_Times/1907/02/24/Sending_Photographs_by_Telegraph}	  
Acesso em: 01 de dezembro de 2025
\end{thebibliography}
}
\vspace{0.2cm}

\vfill

%\pagebreak 

% Mini bios 
% Seja informal e divertido
% Prefira fotos com fundo branco
%\vfill

\begin{wrapfigure}{L}{1.7cm}
	\centering
	\includegraphics[width=2cm]{anapinheiro.jpg}
\end{wrapfigure}\noindent
Ana Pinheiro é professora do Departamento de 
Matemática da UFBA, doutora pela UFRJ, na 
área de Geometria Diferencial. Atualmente, 
cursa o Mestrado em Artes Visuais, na Escola 
de Belas Artes da UFBA, onde estuda Fotografia. 
Quer muito entender o mundo, e desconfia que 
quem não sabe História, não sabe nada.

\end{document}
