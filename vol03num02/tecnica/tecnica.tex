\documentclass{hipatia}
\usepackage{lipsum}
\usepackage{makecell}
\usepackage{mathtools}

\newtheorem{problema}{\noindent \sffamily\textbf{Problema}}
\newcommand{\sol}{\noindent \sffamily \textbf{Solução: }\rmfamily}

%Use \DeclareMathOperator para definir novos
% operadores para o modo matemático
\DeclareMathOperator{\sen}{sen}
%\DeclareMathOperator{\cos}{cos}
\DeclareMathOperator{\cis}{cis}

%Evite numerar teoremas
%Prefira nomeá-los
%Use os ambientes abaixo
\newtheorem*{theorem*}{Teorema}
\newtheorem*{lemma*}{Lema}
\newcommand{\superou}{\textsuperscript{\underline{o}}~}



% Evite títulos muito longos
\title{Equações de 3\superou  grau}
% Se for necessário diminuir a fonte do título
% para caber no quadro, use
% \title{ \fontsize{28}{28}\selectfont Uma Nova Demonstração do\\  \fontsize{28}{28}\selectfont Teorema de Pitágoras}

% O Subtítulo é o nome da seção da revista
% Deve ser uma palavra de origem grega
\subtitle{Técnica}
\author{Armando Barbosa}
% A data não é necessária
%\date{October 2023}
% Não se preocupe com a numeração
% das páginas ou com o número da edição

\begin{document}
\setcounter{page}{\tecnicapage}
\maketitle

\section{Introdução}

A busca por soluções explícitas de equações polinomiais envolvendo radicais é um dos episódios mais fascinantes do desenvolvimento da matemática. Ele se inicia ainda na antiguidade com fórmulas conhecidas dos babilônicos e atinge o seu ápice no século XIX com os trabalhos Abel e Galois.

O objetivo deste artigo é apresentar aplicações das fórmulas de Tartaglia--Cardano, que surgiram no Renascimento Italiano, para solucionar equações de 3\superou  grau em problemas de vestibulares e olimpíadas de matemática. No final, também discutiremos brevemente fórmulas por radicais para equações de graus maiores.

Uma equação do 3\superou  grau 
de coeficientes reais 
na variável $x$
é uma expressão algébrica da forma
$$
ax^3 + bx^2 + cx + d = 0, 
$$
com $a,b,c,d \in \mathbb{R}$  e $a \neq 0$.
O Teorema Fundamental da Álgebra garante que essa equação possui três 
soluções complexas, contando-se multiplicidades. 
Estamos interessados em saber quais são
os três valores complexos de $x$ que
satisfazem a equação acima. Para isso, 
podemos dividir a estratégia
para obter uma fórmula para esses valores
em três partes:

\begin{enumerate}
 \item Eliminar o coeficiente de $x^2$ através
 de uma mudança de variável que crie uma
 nova equação do 3\superou  grau cuja 
 soma das raízes é nula;
 \item Realizar uma nova mudança de variável, 
 reduzindo o problema de determinar 
 uma raiz da equação do 3\superou  grau do passo anterior
 à resolução de uma equação do 
 2\superou grau com raízes $u^3$ e $v^3$;
 \item Perceber que as outras duas soluções, 
 álem de $u+v$,
 conjugadas ou não, são $uw + vw^2$ e 
 $uw^2 +vw$, sendo $w = \cis 120^{\circ}$.
\end{enumerate}

Expliquemos agora 
cada um desses passos
com mais detalhes.

\section{Eliminando o termo quadrático}

Dividindo a equação original por $a$, temos que
$$
x^3 + \dfrac{b}{a} x^2 + \dfrac{c}{a} x + \dfrac{d}{a} = 0.
$$
Pelas relações de Girard, as três soluções $x_1$, $x_2$ e $x_3$ 
somadas resultam em $-\dfrac{b}{a}$. Daí, considerando a 
substituição de variáveis
$$
t = x + \dfrac{b}{3a} \Leftrightarrow x=t - \dfrac{b}{3a},
$$
teremos que
\begin{align*}
t_1 + t_2 + t_3 &= (x_1+\dfrac{b}{3a}) + (x_2+\dfrac{b}{3a}) + (x_3+\dfrac{b}{3a}) \\
&= (x_1 + x_2 + x_3) + 3 \cdot \dfrac{b}{3a} \\
&= -\dfrac{b}{a} + \dfrac{b}{a} = 0,
\end{align*}              
e a soma das raízes na variável $t$ será igual a $0$.
Substituindo, então, na equação do 3\superou  grau, podemos concluir que
\begin{gather*}
 \left( t - \dfrac{b}{3a} \right)^3 + \dfrac{b}{a} \cdot \left( t - \dfrac{b}{3a} \right)^2 + \dfrac{c}{a} \cdot \left( t - \dfrac{b}{3a} \right) + \dfrac{d}{a} = 0 \\
\therefore t^3 - 3t^2 \cdot \dfrac{b}{3a} + 3t \cdot \dfrac{b^2}{9a^2} - \dfrac{b^3}{27a^3} +\\
  t^2 \cdot \dfrac{b}{a} - 2t \cdot  \dfrac{b^2}{3a^2} + \dfrac{b^3}{9a^3}+ \\
 t \cdot \dfrac{c}{a} - \dfrac{bc}{3a^2} + \dfrac{d}{a} = 0 \\
 \therefore t^3 + t \left( \dfrac{b^2}{3a^2} - 2\dfrac{b^2}{3a^2} + \dfrac{c}{a} \right) + \left( - \dfrac{b^3}{27a^3} + \dfrac{b^3}{9a^3} - \dfrac{bc}{3a^2} + \dfrac{d}{a} \right) = 0, \\
\end{gather*}
ou seja,
$$
\boxed{ t^3 + t \left( \dfrac{-b^2+3ac}{3a^2} \right) + \left( \dfrac{2b^3-9abc+27a^2d}{27a^3} \right) = 0, }
$$
e o termo quadrático foi eliminado.


\section{Redução para uma equação quadrática}

Sejam
$$
p = \dfrac{-b^2+3ac}{3a^2} \,\,\,\textrm{e}\,\,\, q = \dfrac{2b^3-9abc+27a^2d}{27a^3}.
$$
Daí, temos que
$$
t^3 + pt + q = 0.
$$
Considerando a nova substituição de variáveis
$$
t = u + v,
$$
%\noindent sendo $u$, $v$ reais que escolheremos em breve.
podemos concluir que
\begin{gather*}
 \left( u + v \right)^3 + p \cdot \left( u + v \right) + q = 0 \\
\therefore u^3 + v^3 + 3uv \left( u+ v \right) + p \cdot \left( u + v \right) + q = 0 \\
\therefore \left( u^3 + v^3 + q \right) + \left( u + v \right) \cdot \left( 3uv + p \right) = 0.
\end{gather*}

\begin{figure}[htb!]
\begin{center}
\includegraphics[width=6cm]{formulasecreta.png}
\end{center}       
\caption{
``Niccolò Tartaglia (à direita) era um professor 
ambicioso que possuía uma fórmula secreta
--- a chave para desvendar um problema matemático
aparentemente insolúvel. Gerolamo Cardano (à esquerda) 
era um médico, erudito brilhante e 
notório jogador que não hesitaria em usar 
bajulação e até mesmo artimanhas 
para descobrir o segredo de Tartaglia.''
\cite{Toscano}
}
\end{figure}


Podemos, então, definir $u$ e $v$ de modo a zerar as duas parcelas acima. Para isso, basta escolher $u$ e $v$ tais que
\begin{equation}
\left\{\begin{array}{rl}
u^3 + v^3 +q&= 0 \\
3uv + p&=0,
\end{array}\right.,\tag{I}
\end{equation}
de modo que obtemos o sistema
\begin{equation}
\left\{\begin{array}{rl}
 u^3 + v^3 &= -q \\
 u^3 v^3 &=  -\dfrac{p^3}{27}
\end{array}\right.,\tag{II}
\end{equation}
em $u^3$ e $v^3$.
Note que toda soluçao de (I)
é uma solução de (II), mas
a recíproca não é necessariamente verdadeira,
se considerarmos $u$ e $v$ complexos.
Ou seja, devemos verificar posteriormente
quais soluções de (II) satisfazem (I).

De todo modo, pelas 
relações de soma e produto, podemos concluir que $u^3$
 e $v^3$ são raízes da equação do 2\superou  grau em $z$ dada por
$$
 z^2-(u^3+v^3)z+u^3v^3 =  0,
$$
ou seja, 
$$
 z^2 + q z - \dfrac{p^3}{27} = 0.
$$
Portanto, temos que
$$
 z = \dfrac{-q \pm \sqrt{q^2 + \dfrac{4p^3}{27}}}{2},
 $$
e definindo  
 $\Delta = \dfrac{q^2}{4} + \dfrac{p^3}{27}$,
temos que 
       $$
       z = \dfrac{-q \pm \sqrt{4\Delta}}{2} = -\dfrac{q}{2} \pm \sqrt{\Delta}.
       $$
       Logo,
       $$
       \left\{\begin{array}{rl}
       u^3 &= -\dfrac{q}{2} + \sqrt{\Delta} \\
       v^3 &= -\dfrac{q}{2} - \sqrt{\Delta}
       \end{array}\right..
       $$ 
Lembrando que 
$x=t-\dfrac{b}{3a}$ e
$$ t = u + v = \sqrt[3]{u^3} + \sqrt[3]{v^3},$$
obtemos finalmente que
$$
\boxed{ x = - \dfrac{b}{3a} + \sqrt[3]{-\dfrac{q}{2} + \sqrt{\dfrac{q^2}{4} + \dfrac{p^3}{27}}} + \sqrt[3]{-\dfrac{q}{2} - \sqrt{\dfrac{q^2}{4} + \dfrac{p^3}{27}}} },
$$
que é a fórmula de Tartaglia--Cardano para uma das raízes da equação do 3\superou  grau.
Mais adiante entenderemos o que acontece
nos casos em que $\Delta > 0$, $\Delta = 0$ e $\Delta < 0$.

\section{Fórmulas para as demais raízes}

Se $\omega = \cis \left( 120^{\circ} \right) = -\dfrac{1}{2} + i \dfrac{\sqrt{3}}{2}$, podemos escrever
$$\left\{\begin{array}{rl}
 \left( u \omega \right)^3 + \left( v \omega^2 \right)^3 &=  u^3 + v^3 \\
 \left( u \omega \right)^3 \cdot \left( v \omega^2 \right)^3 &=  u^3 v^3
\end{array}\right.,$$
já que $\omega^3 = 1$.
Assim o par $u \omega$ e $v \omega^2$ 
satisfaz os mesmo sistemas que $u$ e $v$
(por outro lado, o par $u \omega$ e $v \omega$
satisfaz o sistema (II), mas não o (I),
assim um elemento do par deve ter o $\omega$ e o outro $\omega^2$
para satisfazer (I)).
Logo $u \omega + v \omega^2$ também
 é raiz da mesma equação de 3\superou
grau em $t$ encontrada no passo anterior. 
O mesmo vale para $u\omega^2 + v\omega$, que é o 
conjugado de $u \omega + v \omega^2$ quando $u$ e $v$ 
são reais, pois $\omega^2 = \overline{\omega}$.
Portanto, as três raízes de
 $ax^3 + bx^2 + cx + d = 0$ são da forma
\begin{gather*}
 x_1 = -\dfrac{b}{3a} + \sqrt[3]{-\dfrac{q}{2} + \sqrt{\Delta}} + \sqrt[3]{-\dfrac{q}{2} - \sqrt{\Delta}} \\
 x_2 = -\dfrac{b}{3a} + \sqrt[3]{-\dfrac{q}{2} + \sqrt{\Delta}} \cdot w
 + \sqrt[3]{-\dfrac{q}{2} - \sqrt{\Delta}} \cdot w^2\\
 x_3 = -\dfrac{b}{3a} + \sqrt[3]{-\dfrac{q}{2} + \sqrt{\Delta}} \cdot w^2 
 + \sqrt[3]{-\dfrac{q}{2} - \sqrt{\Delta}} \cdot w
\end{gather*}
onde, 
$$ q = \dfrac{2b^3-9abc+27a^2d}{27a^3} $$
e $$ p = \dfrac{-b^2+3ac}{3a^2}.$$

Assim como ocorre nas equações de 2\superou  grau,
o valor
$$
\Delta = \dfrac{q^2}{4} + \dfrac{p^3}{27}
$$
é muito importante,   
 uma vez que seu sinal
ajuda a entender quando temos raízes reais.
Vamos ver como isso acontece através 
de alguns problemas.

\section{Problemas}

Seja $x_i = -\dfrac{b}{3a} + t_i$. 
Como $-\dfrac{b}{3a}$ é um número real, 
$x_i$ é real se, e somente se, $t_i$ é real. 
Portanto, para encontrar as raízes reais, 
podemos nos ater a estudar apenas equações 
do tipo $t^3 +pt + q = 0$. 

%Além disso, podemos ver que $\Delta > 0$ gera uma raiz real e duas complexas conjugadas.

%Vejamos um exemplo prático disso, atráves de uma questão tradicional:

\begin{problema} Calcule o valor de 
$$\sqrt[3]{20 + 14 \sqrt{2}} + \sqrt[3]{20-14\sqrt{2}}.$$ 
\end{problema}
\sol Considere os números reais $u$ e $v$ tais que
$$
u^3 = 20 + 14 \sqrt{2} \qquad v^3 = 20 - 14\sqrt{2}
$$
Note que $u^3+v^3=40$ e $u^3v^3=400-392=8=2^3$.
Como
$$
\left( u + v \right)^3 = u^3 + v^3 + 3uv \left( u + v \right), $$
se $t=u+v$, que é o valor procurado, obtemos
$$t^3 - 6t - 40 = 0.
$$

Pelo teste da raiz racional, testando os divisores de $-40$, temos que $t=4$ é raiz. Dividindo por $\left( t-4 \right)$, podemos concluir que
\begin{gather*}
 t^3 - 6t - 40 = \left( t - 4 \right) \cdot \left( t^2 + 4t + 10 \right) \\
 t_1 = 4 \qquad t_{2,3} = \dfrac{-4 \pm \sqrt{-24}}{2} = -2 \pm i\sqrt{6}
\end{gather*}

Logo, nossa resposta é $t=4$, pois $-2 \pm i\sqrt{6} \not \in \mathbb{R}$. \hfill $\blacksquare$ \\

Na notação anterior, considerando a equação $t^3 - 6t - 40=0$, temos que
$$
p = -6 ,\quad q = -40$$
e
\begin{eqnarray*}
\Delta & = &  \dfrac{q^2}{4} + \dfrac{p^3}{27} \\
       & = & \dfrac{\left(-40 \right)^2}{4} + \dfrac{\left(-6 \right)^3}{27} \\
       & = & 392 > 0
\end{eqnarray*}

Daí, as raízes $t_2$ e $t_3$ e, consequentemente, 
$x_2$ e $x_3$ não são reais, pois teremos valores
 reais distintos 
 $$\sqrt[3]{-\dfrac{q}{2} + \sqrt{\Delta}}
 \text{ e }
 \sqrt[3]{-\dfrac{q}{2} - \sqrt{\Delta}}$$
 multiplicados por 
 $w = -\dfrac{1}{2} + i \dfrac{\sqrt{3}}{2} $
  e por $ w^2 = -\dfrac{1}{2} - i \dfrac{\sqrt{3}}{2}$
   na fórmula encontrada no final da seção anterior,
produzindo necessariamente um número com parte
imaginária não nula. 

%Percebamos também que um termo $i\sqrt{6}$ faz sentido juntando o $i\sqrt{3}$ com o $\sqrt{\Delta} = \sqrt{392} = 14\sqrt{2}$.

Vejamos agora um exemplo com $\Delta < 0$ ao trocarmos o $-40$ 
por $+4$ na última equação:  \\

\begin{problema} Determine todas as soluções de 
       $t^3 - 6t + 4 = 0$. \end{problema}

\sol Pelo teste da raiz racional, testando os divisores 
de $+4$, temos que $t_1 = 2$ é raiz.

Dividindo por $\left( t-2 \right)$, 
podemos concluir que
\begin{gather*}
 t^3 - 6t + 4 = \left( t - 2 \right) \cdot \left( t^2 + 2t - 2 \right) \\
 t_1 = \boxed{ 2 } \qquad t_{2,3} = \dfrac{-2 \pm \sqrt{12}}{2} = \boxed{ -1 \pm \sqrt{3} }
\end{gather*} \hfill $\blacksquare$

Repetindo a analise anterior para 
$t^3 - 6t + 4=0$, temos que
$$
p = -6 ,\quad q = 4$$
e 
\begin{eqnarray*}
\Delta & = & \dfrac{q^2}{4} + \dfrac{p^3}{27} \\
       & = & -4.
%= \dfrac{\left(4 \right)^2}{4} + \dfrac{\left(-6 \right)^3}{27} = -4 < 0
\end{eqnarray*}

% \Rightarrow \dfrac{q^2}{4} + \dfrac{p^3}{27} = \dfrac{\left(4 \right)^2}{4} + \dfrac{\left(-6 \right)^3}{27} = -4 < 0


Calculemos $t_1$ pela fórmula, lembrando que $-2 \pm 2i = \sqrt{8} \cdot \cis \left( \pm 135^{\circ} \right)$:
\begin{eqnarray*}
 t_1 & = &  \sqrt[3]{-\dfrac{q}{2} + \sqrt{\Delta}} + \sqrt[3]{-\dfrac{q}{2} - \sqrt{\Delta}} \\
& = & \sqrt[3]{ -2 + 2i} + \sqrt[3]{ -2 - 2i} \\
& = & \sqrt{2} \cdot \left( \cis \left( 45^{\circ} \right) + \cis \left( - 45^{\circ} \right) \right) \\
& = &  2\sqrt{2} \cdot \cos \left( 45^{\circ} \right) \\
& = & 2. 
\end{eqnarray*}


Calculando agora $t_2$:
\begin{eqnarray*}
 t_2 & = & \sqrt[3]{ -2 + 2i} \cdot w + \sqrt[3]{ -2 - 2i} w^2 \\
  & = & \sqrt[3]{ -2 + 2i} \cdot \left( -\dfrac{1}{2} + i \dfrac{\sqrt{3}}{2} \right) + \sqrt[3]{ -2 - 2i} \left( -\dfrac{1}{2} - i \dfrac{\sqrt{3}}{2} \right).
\end{eqnarray*}

Analisando a expressão acima, é difícil 
acreditar que $t_2$ é um número real. Porém, 
podemos simplificá-la:
\begin{eqnarray*}
 t_2 & = & \sqrt{2} \cdot \cis \left( 45^{\circ} \right)  \cdot w + \sqrt{2} \cdot \cis \left( -45^{\circ} \right)  \cdot w^2 \\
 & = & \sqrt{2} \cdot \cis \left( 45^{\circ} \right)  \cdot w + \sqrt{2} \cdot \overline{\cis \left( 45^{\circ} \right)}  \cdot \overline{w} \\
 & = & \sqrt{2} \cdot \left( 2\text{Re}\,\left( \cis 45^{\circ} \cdot  w \right) \right)\\
 & = & 2 \sqrt{2} \cdot \cos \left( 165^{\circ} \right) \\
% & = & - 2\sqrt{2} \cdot \cos \left( 15^{\circ} \right) \\
% & = & - 2\sqrt{2} \cdot \left( \dfrac{\sqrt{6}}{4} + \dfrac{\sqrt{2}}{4} \right) \\
 & = & -\sqrt{3} - 1  \in \mathbb{R}.
\end{eqnarray*}
De modo análogo, podemos
 obter $t_3 = \sqrt{3} - 1  \in \mathbb{R}$.


%calcular $t_3$ só para conferir os resultados:
%\begin{gather*}
% t_3 = \sqrt{2} \cdot \cis \left( 45^{\circ} \right)  \cdot \cis \left( 240^{\circ} \right) + %\sqrt{2} \cdot \cis \left( -45^{\circ} \right)  \cdot \cis \left( 120^{\circ} \right) \\
% = \sqrt{2} \cdot \left( \cis \left( 285^{\circ} \right) + \cis \left( 75^{\circ} \right) %\right) = 2 \sqrt{2} \cdot \cos \left( 75^{\circ} \right) \\
% =  \boxed{ \sqrt{3} - 1 } \in \mathbb{R}
%\end{gather*}

Isso é o que ocorre com qualquer equação 
cúbica com $\Delta$ negativo,
pois 
$$-\dfrac{q}{2} + \sqrt{\Delta}
\text{ e }
-\dfrac{q}{2} - \sqrt{\Delta},$$
são números complexos conjugados,
 cujas raízes cúbicas também são
 conjugadas. Assim, ao multiplicá-las por
 $w$ e $w^2=\overline{w}$, respectivamente,
as partes imaginárias se cancelam e 
obtemos três soluções reais. 
O caso $\Delta = 0$ gera $t_2 = t_3$, 
sendo um número real, pois cancelam-se 
as partes imaginárias. 

Em resumo,
obtemos o seguinte resultado geral:
\begin{center}
 \begin{tabular}{|c|c|}
    \hline
        $ \Delta > 0$ & \makecell{ 1 raiz real e 2 raízes \\ complexas conjugadas} \\
    \hline
    $ \Delta < 0$ & 3 raízes reais e distintas \\
    \hline
    $ \Delta = 0$ & \makecell {3 raízes reais, sendo \\ 
    pelo menos 2 iguais } \\
    \hline
 \end{tabular}
\end{center}

\mbox{ } \\

%\noindent lembrando que, para $ax^3 + bx^2 + cx +d = 0$, temos que:
%$$
%p = \dfrac{-b^2+3ac}{3a^2} \qquad q = \dfrac{2b^3-9abc+27a^2d}{27a^3}
%$$

%Para treinar um pouco, façamos uma questão do IME:

\begin{problema} \emph{(IME/2025)} A equação 
$x^3 - \alpha x + \beta= 0$, onde $\alpha$ e $\beta$ 
são constantes reais, admite raiz não real de módulo 
$\gamma$. Determine $\alpha$ em função de $\beta$ e $\gamma$.
\end{problema}

\sol Se admite raiz não real $g$, conforme vimos anteriormente considerando $x = u + v$, temos que

\begin{gather*}
 (I) \begin{cases}
  u^3 + v^3 = -\beta \\
  3uv = \alpha
 \end{cases} \qquad \begin{cases} \qquad g = u \omega + v \omega^2 \\
 \omega = \cis \left( 120^{\circ} \right) \qquad \gamma = \left| g \right|  \end{cases} \\
 \Rightarrow \gamma^2 = \left[ \left( - \dfrac{1}{2} \right) \cdot \left( u + v \right) \right]^2 + \left[ \left(  \dfrac{\sqrt{3}}{2} \right) \cdot \left( u - v \right) \right]^2  \\
 \gamma^2 = \dfrac{1}{4} \cdot \left[ \left( u + v \right)^2 + 3 \left( u - v \right)^2 \right] \Rightarrow \boxed{ \gamma^2 = u^2 + v^2 - uv } \quad (II) \\
 \gamma^2 = u^2 + v^2 - uv = \left( u + v \right)^2 - 3uv \\
 \xRightarrow{(I)} u^3 + v^3 = -\beta \xRightarrow{(II)} \left( u + v \right) = \dfrac{-\beta}{\gamma^2} \\
 \xRightarrow{(I)} \gamma^2 = \left( u + v \right)^2 - 3uv = \left(\dfrac{-\beta}{\gamma^2} \right)^2 - \alpha \Rightarrow \boxed{ \alpha = \dfrac{\beta^2}{\gamma^4} - \gamma^2 }
\end{gather*} \hfill $\blacksquare$ \\

\begin{problema} Encontre o valor de $\sen 18^{\circ}$.
\end{problema}
\sol 
A partir da identidade $ \sen 3x = 3 \sen x - 4\sen^3 x$ podemos concluir que 

\begin{eqnarray*}
4 \sen^3 18^{\circ} -3 \sen 18^{\circ} & = &  - \sen 54^{\circ} \\
                                       & = & - \cos 36^{\circ} \\
                                       & = & -1 + 2 \sen^2 18^{\circ}
\end{eqnarray*}
Se  $x = \sen 18^{\circ}$, temos $4x^3 -2x^2   - 3x +1 = 0$. Como 
 $x = 1$ é raiz, segue que $4x^3 -2x^2   - 3x +1 = \left( x - 1 \right) \cdot \left( 4x^2 +2x -1 \right) =0$. Daí, como $ \sen 18^{\circ} \neq 1$, temos $\sen 18^{\circ} = \dfrac{-1 \pm \sqrt{5}}{4}$. Dado que $\sen 18^{\circ} > 0$, obtemos $\sen 18^{\circ} = \dfrac{\sqrt{5}-1}{4} $.


\section{Considerações sobre equações de graus maiores}

Para equações do 4\superou  grau, também existe um método para a obtenção de fórmulas para as raízes por meio de radicais conhecido como método de Ferrari. Para equações de grau $\geqslant 5$, tem-se o teorema de Abel-Ruffini que afirma que não há uma solução geral através de operações com radicais, incluindo soma, subtração, multiplicação e divisão envolvendo os coeficientes da equação polinomial.

Notemos que isso não significa que é impossível resolver qualquer uma dessas equações de grau $\geqslant 5$. Por exemplo, a equação do 6\superou  grau
$$
x^6 - 9x^3 + 8 = 0
$$
\noindent pode ser completamente resolvida e reduzida para um caso já estudado realizando a troca de $x^3$ por $y$. A nova equação possui raízes $y=1$ e $y=8$. Daí,
$$
\Rightarrow \begin{cases}
             y =x^3 = 1 \Rightarrow x = 1, \omega, \omega^2 \\
             y = x^3 = 8 \Rightarrow x = 2, 2\omega, 2\omega^2
            \end{cases}
$$

Além disso, por vezes, podemos adaptar as ideias apresentadas para situações parecidas. Por exemplo, para a equação do 5\superou  grau
$$
x^5 + px^3 + \dfrac{p^2}{5} x + q = 0,
$$
sendo $p$ um valor dado, podemos fazer a mudança $x = u + v$ para obter que
\begin{gather*}
 \left( u + v \right)^5 + p \cdot \left( u + v \right)^3 + \dfrac{p^2}{5} \cdot \left( u + v \right) + q = 0 \\
 \therefore \left( u^5 + v^5 \right) + 5uv \left( u + v \right)^3 - 5u^2v^2 \left( u + v \right) + p \left( u + v \right)^3 \\
 + \dfrac{p^2}{5} \left( u + v \right) + q = 0 \\
 \Rightarrow \begin{cases}
              u^5 + v^5 = -q \\
              5uv = -p \Rightarrow u^5 v^5 = -\dfrac{p^5}{5^5}
             \end{cases} \\
             \Rightarrow u^5, v^5 \text{ são raízes de } z^2 + qz - \dfrac{p^5}{5^5} = 0
\end{gather*}

\begin{problema} 
       \emph{(Bulgária/2023 - Outono)} Encontre todas as soluções da equação
$$
 \left( x+1 \right) \sqrt{x^2+2x+2} + x \sqrt{x^2+1} = 0.
$$\end{problema}

\sol Passando o segundo termo para o outro lado e elevando ao quadrado, temos que
$$
 \left( x+1 \right)^2 \left[ \left( x + 1 \right)^2 + 1 \right] = x^2 \left( x^2 + 1 \right)
$$

A expressão $t \left( t +1 \right)$ é crescente para $t \geqslant 0$.

Logo, temos que
$$
\left( x + 1 \right)^2 = x^2 \Rightarrow \boxed{ x = -\dfrac{1}{2} }
$$

\noindent que é a única solução. \hfill $\blacksquare$ \\

Para finalizar, deixaremos duas questões de equações de 3\superou  grau de treino para o leitor.

\begin{problema}
Determine todas as raízes das equações do 3\superou  grau a seguir:
\begin{enumerate}
 \item $t^3 - 6t - 9=0$;
 \item $t^3 - 6t - 4=0$.
\end{enumerate}
\end{problema}

\begin{problema}
\emph{(Hong Kong/2014 - adaptada)} Determine o valor simplificado de $\left( \sqrt[3]{\sqrt{5}+ 2} + \sqrt[3]{\sqrt{5}-2} \right)^{2014}$.
\end{problema} 
\addtocounter{problema}{-2}

\section{Respostas}

\begin{problema} 
 \begin{enumerate}
 \item $3$ e $-\dfrac{3}{2} \pm \dfrac{\sqrt{3}}{2}i$;
 \item $-2$ e $1 \pm \sqrt{3}$.
\end{enumerate}
\end{problema} 


\begin{problema}  $5^{1007}$. Uma ideia é fazer $a = \sqrt[3]{\sqrt{5}+ 2}$, $b= \sqrt[3]{\sqrt{5}- 2}$ e $x= a + b$ para chegar a
$$
x^3 = a^3 + b^3 + 3ab \left( a + b \right) = 2\sqrt{5} + 3x.
$$\end{problema} 

Daí, basta perceber que $x = \sqrt{5}$ é raiz e as outras raízes são complexas, podendo só dividir por $\left( x - \sqrt{5} \right)$ para concluir isso.

Portanto, $x = \sqrt{5}$ e $x^{2014} = \boxed{ 5^{1007} }$.

%\pagebreak

\begin{thebibliography}{99}
\bibitem{Toscano} Toscano, Fábio. \textit{A Fórmula Secreta}. Editora Unicamp,  São Paulo, 2012.
\end{thebibliography}

\vspace{0.1cm}

\begin{wrapfigure}{L}{1.7cm}
\includegraphics[width=2cm]{armando.jpg}
\end{wrapfigure}  
\noindent José Armando Barbosa Filho é iteano, 
ex-olímpico e trabalha olimpíadas de matemática 
desde 2012. Os principais destaques de sua carreira 
são os livros da coleção IME/ITA/Cone Sul/EGMO, 
da qual se orgulha de ser o autor, e ter sido 
vice-líder da equipe do Brasil na IMO/2018. 
Atualmente, está vivendo os primeiros dias de 
pai do José Heitor. Muito \emph{nerd} a ponto 
de ser fã do personagem Leonard Hofstadter, da 
série \emph{The Big Bang Theory}.

\vspace{0.5cm}


\end{document}


